\documentclass{llncs}

\usepackage[a4paper,includeheadfoot,top=1.65in,bottom=1.65in,left=1.73in,hcentering]{geometry}
\usepackage[pdftex]{graphicx}

\usepackage[utf8]{inputenc}

\usepackage[hungarian]{babel}
\selectlanguage{hungarian} 

\begin{document}

\pagestyle{myheadings}
\def\leftmark{{\rm IX. Magyar Sz\'am\'\i t\'og\'epes Nyelv\'eszeti Konferencia}}
\def\rightmark{{\rm Szeged, 2013. január 7-8.}}

\setcounter{page}{3}

\title{\ \break Automatikusan tulajdonnév-annotált korpuszok előállítása a Wikipédiából}
\author{Nemeskey Dávid Márk\inst{1}, Simon Eszter\inst{2}}
\institute{
\inst{}%
MTA SZTAKI \break
1111 Budapest, Lágymányosi utca 11., e-mail:nemeskey@sztaki.hu \break
\and
\inst{}%
MTA Nyelvtudományi Intézet \break
1068 Budapest, Benczúr u. 33., e-mail: simon.eszter@nytud.mta.hu}

\maketitle

\section{Introduction}

\section{Wikipedia and NER}

\section{Creating the Corpus}
\subsection{Creating the English Corpus}

\section{Data description}

\section{Evaluation}

\section{Conclusion}

\cite{Nothman:08} \cite{Szarvas:06} \cite{Medelyan:09}

%
% ---- Bibliography ----
%
\bibliographystyle{huplain}
\bibliography{abstract}

\end{document}
