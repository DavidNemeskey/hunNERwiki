\documentclass{llncs}

\usepackage[a4paper,includeheadfoot,top=1.65in,bottom=1.65in,left=1.73in,hcentering]{geometry}
\usepackage[pdftex]{graphicx}

\usepackage[utf8]{inputenc}

\usepackage[hungarian]{babel}
\selectlanguage{hungarian} 

\begin{document}

\pagestyle{myheadings}
\def\leftmark{{\rm IX. Magyar Sz\'am\'\i t\'og\'epes Nyelv\'eszeti Konferencia}}
\def\rightmark{{\rm Szeged, 2013. január 7-8.}}

\setcounter{page}{3}

\title{\ \break Automatikusan tulajdonnév-annotált korpuszok előállítása a Wikipédiából}
\author{Nemeskey Dávid Márk\inst{1}, Simon Eszter\inst{2}}
\institute{
\inst{}%
MTA SZTAKI \break
1111 Budapest, Lágymányosi utca 11., e-mail:nemeskey@sztaki.mta.hu \break
\and
\inst{}%
MTA Nyelvtudományi Intézet \break
1068 Budapest, Benczúr u. 33., e-mail: simon.eszter@nytud.mta.hu}

\maketitle

\section{Introduction}

Az automatikus tulajdonnév-felismerés (Named Entity Recognition, NER) a természetes nyelv feldolgozását célzó alkalmazások közül az egyik legnépszerűbb, mivel hatékonyan automatizálható, és eredménye hasznos bemenete különböző magasabb szintű információkinyerő és -feldolgozó rendszereknek. A feladat során strukturálatlan szövegben kell azonosítani és az előre definiált osztályok valamelyikébe besorolni a neveket. Az egyik leggyakrabban alkalmazott nemzetközi sztenderd a CoNLL-2003 verseny kiírását követi, amely szerint a négy névkategória a következő: személy-, intézmény-, földrajzi név és az ezek egyikébe sem tartozó egyéb kategória. 

%Named Entity Recognition (NER), the task of identifying Named Entities (NEs) in unstructured texts and classifying them into pre-selected classes, is one of the most important subtasks in many NLP tasks, such as information retrieval, information extraction or machine translation. MUC and CoNLL datasets and annotation schemes have been the major standards applied in the field of NER. 

A gold standard adathalmazok 

The standard datasets are highly domain-specific (mostly newswire) and are
restricted in size. Researchers attempting to merge these datasets to get a bigger training corpus are faced with the
problem of combining different tagsets and annotation schemes. Manually
annotating large amounts of text with linguistic information is a
time-consuming, highly skilled and delicate job, but large, accurately
annotated corpora are essential for building robust supervised machine
learning NER systems. Therefore, reducing the annotation cost is a key
challenge.

One approach is to generate the resources automatically, another one is to use
collaborative annotation and/or collaboratively constructed resources, such as
Wikipedia, Wiktionary, Linked Open Data, or DBpedia. In this paper we combine
these approaches by automatically generating freely available NE tagged
corpora from Wikipedia.

\section{Wikipedia and NER}

Wikipedia (WP, see {\tt http://wikipedia.org}), a free
multilingual Internet encyclopedia, written collaboratively by volunteers, is
a goldmine of information.
WP has been applied to several NLP tasks such as word sense
disambiguation, ontology and thesaurus building, and question answering (see
Medelyan et al.~\cite{Medelyan:09} for a survey). It is recognized as one
of the largest available collections of entities, and also as a resource that
can improve the accuracy of NER. The most obvious utilization of WP for NER is
extracting gazetteers containing person names, locations or organizations
(e.g.~Toral and Mu\~noz \cite{Toral:06}). Creating dictionaries of
entities is also a common step of NE disambiguation
\cite{Bunescu:06,Cucerzan:07}. Both supervised and unsupervised NER systems
use such lists, see e.g.~Nadeau et al.~\cite{Nadeau:06} The knowledge
embodied in WP may also be incorporated in NER learning as features,
e.g.~Kazama and Torisawa \cite{Kazama:07} showed that automatic
extraction of category labels from WP improves the accuracy of a supervised NE
tagger.
Another approach to improve NER with WP is the automatic creation of training
data. Nothman et al.~\cite{Nothman:08} used a similar method to create
a NE annotated text in English. They transformed the WP links into NE
annotations by classifying the target articles into standard entity
classes. Their approach to classification is based primarily on category head
nouns and the opening sentences of articles where definitions are often given.

Our approach to recognize and classify NEs in corpora generated from WP was to
map the DBpedia ontology classes to standard NE tags and assign these to WP
entities. the one presented here is the first automatically NE annotated corpus for Hungarian.

\section{Creating the English Corpus}

\section{Creating the Hungarian Corpus}

\section{Data description}

\section{Evaluation}

\section{Conclusion}


%
% ---- Bibliography ----
%
\begin{thebibliography}{15}
%
\bibitem{alexin:eacl2003}
Alexin Z., Csirik, J., Gyim\'othy, T., Bibok K., Hatvani, Cs.,
Pr\'osz\'eky, G., Tihanyi, L.:
Manually Annotated Hungarian Corpus.
in Proc. of the Research Note Sessions of the 10th Conference
of the European Chapter of the Association for Computational
Linguistics EACL'03, Budapest, Hungary, 53--56 (2003).
%
\bibitem{bibok:mszny2003}
Bibok K.:
A sz\'or\'ol \'es a sz\'ofajokr\'ol (a sz\'am\'\i t\'og\'epes nyelvfeldolgoz\'as kap- cs\'an),
Magyar Sz\'am\'\i t\'og\'epes Nyelv\'eszeti Konferencia (MSZNY 2003),
Szeged, Magyarorsz\'ag, 31--36, (2003).
%
\bibitem{farkas:mszny2004}
Farkas R., Konczer K., Szarvas Gy.:
Szemantikus keretilleszt\'es \'es az IE rendszer automatikus ki\'ert\'ekel\'se
Magyar Sz\'am\'\i t\'og\'epes Nyelv\'eszeti Konferencia (MSZNY 2004),
bek\"uldve, Szeged, Magyarorsz\'ag, (2004).
%
\bibitem{hocza:acta2004}
H\'ocza, A.:
Noun Phrase Recognition with Tree Patterns
elfogadva az Acta Cybernetica c. lapban t\"ort\'en\H{o} megjelen\'esre (2004).
%
\bibitem{mihaczi:mszny2003}
Mih\'aczi Andr\'as, N\'emeth L\'aszl\'o, R\'acz Mikl\'os:
Magyar sz\"ovegek tem\'eszetes nyelvi el\H{o}feldolgoz\'asa
Magyar Sz\'am\'\i t\'og\'epes Nyelv\'eszeti Konferencia (MSZNY 2003),
Szeged, Magyarorsz\'ag, 38--43, (2003).
%
\bibitem{proszeky:mszny2003}
Pr\'osz\'eky G.:
Automatikus inform\'aci\'oszerz\'es gazdas\'agi r\"ovidh\'\i rekb\H{o}l.
Magyar Sz\'am\'\i t\'og\'epes Nyelv\'eszeti Konferencia (MSZNY 2003),
Szeged, Magyarorsz\'ag, 161--166, (2003).
%
\bibitem{proszeky:neumann2003}
Pr\'osz\'eky G.:
Automatikus inform\'aci\'oszerz\'es gazdas\'agi-politikai r\"ovidh\'\i rekb\H{o}l.
VIII. Orsz\'agos (Centen\'ariumi) Neumann Kongresszus kiadv\'anya,
Budapest, Magyarorsz\'ag, 359--367, (2003).
%
\end{thebibliography}
\end{document}
