% $Header: /home/vedranm/bitbucket/beamer/solutions/generic-talks/generic-ornate-15min-45min.en.tex,v 90e850259b8b 2007/01/28 20:48:30 tantau $

\documentclass[utf8x,t]{beamer}

\mode<presentation>
{
%  \usetheme[height=7mm]{Rochester}
  \usetheme{Rochester}
  \useinnertheme{rounded}
%  \useoutertheme{infolines}
  % or ...

  \setbeamercovered{transparent}
}
\usefonttheme[onlymath]{serif}

\usepackage[magyar]{babel}
\usepackage[utf8x]{inputenc}

\usepackage{times}
\usepackage[T1]{fontenc}
% Or whatever. Note that the encoding and the font should match. If T1
% does not look nice, try deleting the line with the fontenc.

%% to draw trees without jpg figures
%\usepackage{synttree}
% For code blocks
\usepackage{algorithm}  % Conflict w/ synttree
\usepackage{algorithmic}
% for sane tabular handling
\usepackage{array}
% For strikethrough (normalem keeps it as emph)
\usepackage[normalem]{ulem}

\usepackage{mathrsfs}
\usepackage{bm}  % For bold math

\usepackage{booktabs}

\newcommand{\vitem}{\item \vspace{4pt}}
\newcommand{\nyil}{$\rightarrow$\ }
\newcommand{\nagytilde}{$\sim$}
\newtheorem{Examplee}{Example}[section]

\newtheorem{nix}{}[section]

\title % (optional, use only with long paper titles)
{Automatikus korpuszépítés tulajdonnév-felismerés céljára}

%\subtitle
%{Presentation Subtitle} % (optional)

\author % (optional, use only with lots of authors)
{Nemeskey Dávid Márk\inst{1}, Simon Eszter\inst{2}}

\institute{\inst{1} MTA SZTAKI \\
\inst{2} MTA Nyelvtudományi Intézet} % (optional, but mostly needed)
% - Use the \inst command only if there are several affiliations.
% - Keep it simple, no one is interested in your street address.

\date % (optional)
{2013.01.07. \\ MSZNY 2013}

% \subject{Talks}
% This is only inserted into the PDF information catalog. Can be left
% out. 



% If you have a file called "university-logo-filename.xxx", where xxx
% is a graphic format that can be processed by latex or pdflatex,
% resp., then you can add a logo as follows:

% \pgfdeclareimage[height=0.5cm]{university-logo}{university-logo-filename}
% \logo{\pgfuseimage{university-logo}}

%\AtBeginSection[]
%{
%   \begin{frame}
%       \frametitle{Az előadás vázlata}
%       \tableofcontents[currentsection]
%   \end{frame}
%}


% If you wish to uncover everything in a step-wise fashion, uncomment
% the following command: 
%\beamerdefaultoverlayspecification{<+->}


\begin{document}

% TODO: kell outline?
\begin{frame}{}
  \titlepage
\end{frame}

\section{Bevezetés}

\begin{frame}{A feladat és a probléma}

\bigskip

A feladat: a szövegben felismerni és előre definiált osztályok valamelyikébe besorolni a neveket 

\bigskip

A probléma:
\begin{itemize}
\item a sztenderd adathalmazok korlátozott méretűek és témaspecifikusak 
\item robusztus rendszerek építéséhez nagyméretű és heterogén tulajdonnév-annotált korpuszokra van szükség
\item a kézi annotálás rendkívül költséges és erőforrás-igényes
\end{itemize}

\end{frame}

\begin{frame}{A kihívás}

az annotációs költség csökkentése \nyil automatikusan annotált korpuszok előállítása

\begin{itemize}
\item meglévő korpuszok egyesítése \nyil az annotációs sémák és címkekészletek különbözősége 
\item rendelkezésre álló korpuszon tanított tagger futtatása új szövegen \nyil a domének közötti különbség teljesítménycsökkenést okoz
\item közösségi tartalmak felhasználása \nyil leképezés a sztenderd címkekészletekre
\end{itemize}

\smallskip

\begin{nix}
A fenti megközelítések kombinációja: \\ automatikus eszközökkel tulajdonnév-annotált korpuszok építése a Wikipédiából
\end{nix}

\end{frame}

\section{Wikipédia és tulajdonnév-felismerés}

\begin{frame}{Wikipédia és NER}

\bigskip

\begin{nix}
mivel a Wikipédia címszavainak nagy része tulajdonnév, kiválóan használható tulajdonnév-felismeréshez
\end{nix}

\bigskip

Alkalmazási módjai:
\smallskip
\begin{itemize}
\item nagyméretű névlisták előállítása 
\item a Wikipédia kategóriacímkéinek kinyerése és beépítése NER rendszerekbe
\item tanító korpusz előállítása a Wikipédia szövegbázisából
\begin{itemize}
\item az inherens kategóriastruktúra felhasználásával
\item a szócikkek első mondatának felhasználásával
\end{itemize} 
\end{itemize}

\end{frame}

\begin{frame}{Az általunk alkalmazott módszer}

\begin{nix}
A DBpedia ontológiai osztályait képeztük le a CoNLL-névosztályokra, majd ezeket Wikipédia-entitásokhoz kötöttük
\end{nix}

\bigskip

Nyelvfüggetlen módszer \nyil minden Wikipédia-nyelvre építhető automatikusan névannotált korpusz

\bigskip

\begin{description}
\item[Magyar:] ez az első automatikusan névannotált korpusz
\item[Angol:] a \textit{Semantically Annotated Snapshot of English Wikipedia} mellett ez az egyetlen
\end{description}

\end{frame}

\section{Korpuszépítés}

\subsection*{A módszer}

% TODO: bigskip?
\begin{frame}{Korpuszépítés}
  \begin{itemize}
  \vitem A Wikipédia két feltételezett tulajdonsága:
    \begin{enumerate}
    \vitem a szócikkek nagy része entitásokról szól
    \vitem a kereszthivatkozások azonosítják a tulajdonneveket a szövegben
    \end{enumerate}
  \vitem Módszerünk hasonló Nothman et al.~(2008)-hoz:
    \begin{enumerate}
    \vitem a Wikipédia-cikkeket entitásosztályokba soroljuk;
    \vitem a cikkeket mondatokra bontjuk;
    \vitem felcímkézzük a tulajdonneveket a szövegben;
    \vitem kiszűrjük a rossz minőségű mondatokat.
    \end{enumerate}
  \end{itemize}
\end{frame}

\subsection{Cikkek mint entitások}
\begin{frame}{Cikkek mint entitások}
  \bigskip
  Cikkek entitásosztályokba sorolása: % TODO what?
  \smallskip
  \begin{itemize}
  \vitem Szokásos megoldás: felügyelt tanulás (infoboxok, első mondat)
  \vitem A mi megközelítésünk: entitás lista a DBpedia alapján
    \begin{itemize}
    \vitem OWL ontológia kb. 320 osztállyal: kézileg leképeztük CoNLL (Szeged) kategóriákra
    \vitem Minden entitást a legspecifikusabb osztály szerinti kategóriába raktunk
    \end{itemize}
  \item Csökkent fedés, jobb pontosság
  \end{itemize}
\end{frame}

\subsection{Tulajdonnevek címkézése}
\begin{frame}{Tulajdonnevek címkézése}
  \bigskip
  \begin{itemize}
  \vitem Tulajdonnevek felismerése szövegben:
    \begin{itemize}
    \vitem Kereszthivatkozások, ha a hivatkozott cikk szerepel az entitáslistában
    \vitem Az aktuális cikk címe, ha szerepel a listában
    \end{itemize}
  \vitem Címkézés: a DBpedia--Szeged leképezés alapján
  \vitem Kivételek:
    \begin{itemize}
    \vitem Cikkek, amik nem entitások: nagybetűvel kell kezdődjön a szócikk
    \vitem Jelöletlen entitások: megpróbáljuk őket korábbi tulajdonnevekhez kötni
    \vitem Speciális esetek: tulajdonnév-köznév összetételek, mondatkezdő főnevek, stb.
    \end{itemize}
  \end{itemize}
\end{frame}

\subsection{Mondatok szűrése}
\begin{frame}{Mondatok szűrése}
  \bigskip
  Csak a teljesen felcímkézett mondatokat tartjuk meg. Eldobjuk:
  \smallskip
  \begin{itemize}
  \vitem azokat a mondatokat, amiben címkézetlen hivatkozások vagy nagybetűs szavak vannak
  \vitem a mondattöredékeket:
    \begin{itemize}
    \vitem amikben nincs mondatvégi írásjel %és amik nem nagybetűvel kezdődnek
    \vitem állítmány nélküli mondatok %TODO: tényleg? SE: ezeket végül nem dobtuk el, csak leírtuk, hogy el lehet
    \end{itemize}
  \end{itemize}
\end{frame}

\begin{frame}{Nyelvfüggő elemek}
  \bigskip
  \begin{itemize}
  \vitem Nyelvi eszközök
    \begin{itemize}
    \vitem kidolgoztunk egy keretrendszert, amibe bármilyen eszközt be lehet kötni
    \vitem tokenizáló, lemmatizáló, morfológiai egyértelműsítő
    \end{itemize}
  \vitem Wikipédia
    \begin{itemize}
    \vitem Sablonok nyelvfüggőek
    \vitem a magyar egy közepes méretű nyelv: \nagytilde 200k cikk a kb.~\nagytilde 4M angollal szemben
    \end{itemize}
  \vitem DBpedia
    \begin{itemize}
    \vitem nagyobb nyelvekre honosított fejezetek --- magyarra nincs
    \vitem csak az angol Wikipédiában is szereplő magyar tulajdonneveket tartalmazza
%    \vitem távlati terv a lista kiegészítése gépi tanulással  % TODO "Future work"
    \end{itemize}
  \end{itemize}
\end{frame}

\section{Problémás esetek, hibaelemzés}

\begin{frame}{Kézi ellenőrzés}

\bigskip

A teljes korpusz 0,001\%-át kézzel annotáltuk, majd összehasonlítottuk a gépi és kézi annotálás eredményét.

\bigskip

\begin{center}
\begin{tabular}{lcccc}
\toprule
& P (\%) &  R (\%) & F$_{\beta=1}$ (\%) & NE (\#) \\
\midrule
LOC     &   98.72 &  95.65 &  97.16 & 161 \\
MISC    &   95.24 &  76.92 &  85.11 & 26 \\
ORG     &   89.66 &  89.66 &  89.66 & 29 \\
PER     &   88.30 &  89.25 &  88.77 & 93 \\
\midrule
Összesítve &   94.33 &  91.59 &  92.94 & 309 \\
\bottomrule
\end{tabular} 
\end{center}

\end{frame}

\begin{frame}{Igazságmátrix}

\bigskip

\begin{center}
\begin{tabular}{l@{\hspace{0.5em}}|@{\hspace{0.5em}}cccc}
\toprule
Auto$\downarrow$ / Gold$\rightarrow$ & PER & ORG & LOC & MISC \\
\midrule
PER & 83 & 1 & & 2 \\
ORG &  & 26 & 1 & 1 \\
LOC &  & 1 & 154 & \\
MISC &  &  & 1 & 20 \\
\bottomrule
\end{tabular}
\end{center}

\bigskip

A típustévesztés okai:

\begin{itemize}
\item a DBpedia típusinformációja helytelen: \\ $[\mbox{MTA}]_{LOC}$
\item a Wikipédia rossz kereszthivatkozásai: \\ $[\mbox{Walt Disney}]_{PER}$ Co.
\end{itemize}

\end{frame}

\begin{frame}{További hibák}

\bigskip

\begin{tabular}{l@{\hspace{0.5em}}cccc}
\toprule
	& PER & ORG & LOC & MISC \\
\midrule
Álpozitív & 1 &  0  &  1  &  0 \\
Álnegatív & 3 &  0  &  5  &  4 \\
Részlegesen felismert nevek & 7 & 1 & 0 & 0 \\
\bottomrule
\end{tabular}

\bigskip

\begin{itemize}
\item Álpozitív: hibásan névként felismert szavak: $[\mbox{Bizánc-ellenes}]_{LOC}$
\item Álnegatív: fel nem ismert nevek: $[\mbox{ókori Róma}]_O$
\item Részlegesen felismert nevek: pontatlan entitáshatárok: $[\mbox{Szent István király}]_{PER}$
\end{itemize}

\end{frame}

%\begin{frame}{The Hungarian Corpus}
%  The algorithm is, in theory, language-agnostic. We tested this property by
%  applying it to a medium-sized language, Hungarian. \\
%  Difference from bigger, well-resourced languages:
%  \begin{itemize}
%  \vitem E % TODO: ide mit?
%  \end{itemize}
%\end{frame}

\section{Kiértékelés}

%\begin{frame}{Silver standard korpuszok}

%\bigskip

%A silver standard korpuszok

%\smallskip

%\begin{itemize}
%\vitem for less resourced languages they can serve as training corpora in lieu of gold standard datasets
%\vitem they can serve as training sets for domains differing from newswire
%\vitem they can be sources of huge entity lists
%\vitem they can be sources of feature extraction
%\end{itemize}

%\end{frame}

\begin{frame}{Kiértékelés}

\bigskip

\begin{itemize}
\item A kiértékeléshez a \texttt{hunner} taggert használtuk az alábbi jegyekkel:
\begin{itemize}
\vitem mondatkezdő- és vég pozíciók
\vitem szóalakon alapuló jegyek
\vitem morfológiai információ
\vitem listajegyek
\end{itemize}
\item Az eredmények kiszámításához a sztenderd CoNLL-módszert alkalmaztuk
\end{itemize}

\bigskip

\end{frame}

\begin{frame}{Korpuszméret és telítettség}

\bigskip

\begin{tabular}{lccc}
\toprule
 & \textbf{huwiki} & \textbf{sűrített huwiki} & \textbf{Szeged NER} \\
\midrule
token & 19.108.027 & 3.512.249  & 225.963 \\
NE & 456.281 & 456.281  & 25.896 \\
telítettség (\%) & 2,38 & 12,99 & 11,46 \\
\bottomrule
\end{tabular}

\bigskip

\begin{nix}
szigorú módszer \nyil kevés, de pontos NE \nyil híg korpusz
\end{nix}

\bigskip

A kiértékeléshez eltávolítottuk a NE-t nem tartalmazó mondatokat.

\end{frame}

\begin{frame}{Kísérletek}

\begin{nix}
a 3,5 millió tokenes korpuszt 90-10\% arányban tanító és kiértékelő halmazra osztottuk
\end{nix}

\bigskip

\begin{itemize}
\item Szeged NER tanítókorpusz -- Szeged NER tesztkorpusz
\item huwiki tanítóhalmaz -- huwiki teszthalmaz
\item huwiki tanítóhalmaz -- Szeged NER tesztkorpusz
\item Szeged NER tanítókorpusz, a Wikipédiából kinyert névlistákkal -- Szeged NER tesztkorpusz
\item Szeged NER tanítókorpusz, a Wikipédia korpuszon tanított modell által felcímkézve, és a címkék extra jegyként hozzáadva -- Szeged NER tesztkorpusz
\end{itemize}

\end{frame}

\begin{frame}{Eredmények}

\begin{center}
\begin{tabular}{lcccc}
\toprule 
\bf tanítás & \bf teszt & \bf P (\%) & \bf R (\%) & \bf F (\%) \\ 
\midrule
Szeged & Szeged & 94,50 & 94,35 & 94,43 \\
huwiki & huwiki & 90,64 & 88,91 &  \textbf{89,76}  \\
huwiki & Szeged & 63,08 & 70,46 & 66,57  \\
Szeged\_wikilisták & Szeged & 95,48 & 95,48 & 95,48  \\
Szeged\_wikitag & Szeged & 95,38 & 94,92 & 95,15 \\
\bottomrule
\end{tabular}
\end{center}

\bigskip

A saját teszthalmazon elért eredmény arról tanúskodik, hogy a korpusz akár önálló gold standard adathalmazként is használható.

\end{frame}

%\begin{frame}{Results -- analysis}

%\begin{itemize}
%\vitem training and testing across different corpora decreases F-measure
%\vitem the NE tagger trained on WP does not achieve as high performance tested against CoNLL test set as one trained on its own train set
%\vitem gazetteer lists extracted from WP and training with extra features given by the model trained on WP improve F-measure
%\vitem F-measures for Hungarian are higher 
%\begin{itemize}
%\vitem the MISC category is more consistent
%\vitem the Szeged NER corpus is highly accurately tagged 
%\end{itemize}
%\end{itemize}
%  \vfill

%\end{frame}

\section{A korpuszok leírása}

\begin{frame}{Multitag formátum}

% TODO: nem fér ki minden oszlop
%SE: kicsit összébb húztam, de így se fér ki. vagy ki kéne szedni belőle a linkes oszlopokat, vagy valahogy kicsinyíteni az egészet
\begin{tabular}{@{}l@{ }l@{ }l@{ }l@{ }l@{ }l@{ }l}
\textbf{token} & \textbf{text/link} & \textbf{WP link} & \textbf{POS} & \textbf{lemma} & \textbf{NE} \\
Neumann & B-link & Neumann János & NOUN & neumann & B-PER & B-PER \\
János & I-link & Neumann János & NOUN & jános & I-PER & I-PER \\
( & text & 0 & PUNCT & ( & O & O \\
1903–1957 & text & 0 & NUM & 1903–1957 & O & O \\
) & text & 0 & PUNCT & ) & O & O \\
matematikai & text & 0 & ADJ & matematikai & O & O \\
szemszögből & text & 0 & NOUN<CAS<ELA{>}> & szemszög & O & O \\
közelíti & text & 0 & VERB<DEF> & közelít & O & O \\
meg & text & 0 & PREV & meg & O & O \\
a & text & 0 & ART & a & O & O \\
kérdést & text & 0 & NOUN<CAS<ACC{>}> & kérdés & O & O \\
\end{tabular}

%\begin{tabular}{llllll}
%\textbf{token} & \textbf{text/link} & \textbf{WP link} & \textbf{POS} & \textbf{lemma} & \textbf{NE} \\
%This & text & 0 &    DT &   This & O \\
%is &   text & 0 &    VBZ &  be &   O \\
%a &    text & 0 &    DT &   a  &    O\\
%list & text & 0 &    NN &   list  & O\\
%of &   text & 0 &    IN &   of &     O\\
%films  & B-link & films &  NNS &  film & O\\
%by &   text & 0 &    IN &   by &     O\\
%year & text & 0 &    NN &   year & O\\
%produced & text & 0 &    VBN &  produce & O\\
%in &   text & 0 &    IN &   in &     O\\
%the &  text & 0 &    DT &   the &   O\\
%country &text & 0 &    NN &   country &O\\
%of &   text & 0 &    IN &   of &     O\\
%South &  B-link & South Korea &  NNP &  South  &B-LOC\\
%Korea &  I-link & South Korea &  NNP &  Korea   & I-LOC\\
%\end{tabular}

\end{frame}

\begin{frame}{A korpusz elérhetősége}
\bigskip
A korpuszok
\begin{itemize}
\vitem a CC-BY-SA 3.0 Unported licensz alatt kerültek publikálásra
\vitem szabadon letölthetőek a \url{http://hlt.sztaki.hu/resources/} honlapról
\vitem szintén elérhetőek a META-SHARE-en keresztül. Utóbbi
  \begin{itemize}
  \vitem egy nyílt rendszer, mely lehetővé teszi a nyelvi erőforrások megosztását
  \vitem létrehozója a META-NET, az Európai Bizottság által alapított nyelvtechnológiai hálózat
  \vitem \url{http://www.meta-net.eu/meta-share}
  \end{itemize}
\end{itemize}

\end{frame}

\section{Összegzés}
\begin{frame}{Összegzés}
  \begin{itemize}
  \vitem módszerünk nagyban csökkenti az annotációs költségeket 
  \vitem bármely Wikipédia-nyelvre alkalmazható \nyil kevés erőforrással rendelkező nyelvekre is előállítható tulajdonnév-annotált korpusz
  \vitem a létrehozott korpuszok több módon is felhasználhatóak a NER rendszerek hatékonyságának növelésére
  \vitem mivel korpuszunk közelíti a gold standard minőséget, akár önálló tanítóhalmazként is használható a sajtó stílustól eltérő szövegekhez
  \vitem az általunk előállított korpuszok szabadon felhasználhatóak
  \vitem a módszerben benne rejlik a lehetőség finomabb tulajdonnév-hierarchiák támogatására is
  \end{itemize}
  \vfill
\end{frame}

\begin{frame}

\bigskip

\bigskip

{\huge Köszönjük a figyelmet!}

\bigskip
\bigskip
\bigskip

e-mail: \\ {\tt nemeskey.david@sztaki.mta.hu \\ simon.eszter@nytud.mta.hu} \\

\bigskip
\bigskip

URL: \\ {\tt http://hlt.sztaki.hu/resources/}


\end{frame}

\end{document}


