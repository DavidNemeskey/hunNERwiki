% $Header: /home/vedranm/bitbucket/beamer/solutions/generic-talks/generic-ornate-15min-45min.en.tex,v 90e850259b8b 2007/01/28 20:48:30 tantau $

\documentclass[utf8x,t]{beamer}

\mode<presentation>
{
%  \usetheme[height=7mm]{Rochester}
  \usetheme{Rochester}
  \useinnertheme{rounded}
%  \useoutertheme{infolines}
  % or ...

  \setbeamercovered{transparent}
}
\usefonttheme[onlymath]{serif}

\usepackage[magyar]{babel}
\usepackage[utf8x]{inputenc}

\usepackage{times}
\usepackage[T1]{fontenc}
% Or whatever. Note that the encoding and the font should match. If T1
% does not look nice, try deleting the line with the fontenc.

%% to draw trees without jpg figures
%\usepackage{synttree}
% For code blocks
\usepackage{algorithm}  % Conflict w/ synttree
\usepackage{algorithmic}
% for sane tabular handling
\usepackage{array}
% For strikethrough (normalem keeps it as emph)
\usepackage[normalem]{ulem}

\usepackage{mathrsfs}
\usepackage{bm}  % For bold math

\newcommand{\vitem}{\item \vspace{4pt}}
\newcommand{\nyil}{$\rightarrow$\ }
\newcommand{\nagytilde}{$\sim$}
\newtheorem{Examplee}{Example}[section]

\newtheorem{nix}{}[section]

\title % (optional, use only with long paper titles)
{Automatikus korpuszépítés tulajdonnév-felismerés céljára}

%\subtitle
%{Presentation Subtitle} % (optional)

\author % (optional, use only with lots of authors)
{Nemeskey Dávid Márk\inst{1}, Simon Eszter\inst{2}}

\institute{\inst{1} MTA SZTAKI \\
\inst{2} MTA Nyelvtudományi Intézet} % (optional, but mostly needed)
% - Use the \inst command only if there are several affiliations.
% - Keep it simple, no one is interested in your street address.

\date % (optional)
{2013.01.07. \\ MSZNY 2013}

% \subject{Talks}
% This is only inserted into the PDF information catalog. Can be left
% out. 



% If you have a file called "university-logo-filename.xxx", where xxx
% is a graphic format that can be processed by latex or pdflatex,
% resp., then you can add a logo as follows:

% \pgfdeclareimage[height=0.5cm]{university-logo}{university-logo-filename}
% \logo{\pgfuseimage{university-logo}}

%\AtBeginSection[]
%{
%   \begin{frame}
%       \frametitle{Az előadás vázlata}
%       \tableofcontents[currentsection]
%   \end{frame}
%}


% If you wish to uncover everything in a step-wise fashion, uncomment
% the following command: 
%\beamerdefaultoverlayspecification{<+->}


\begin{document}

% TODO: kell outline?
\begin{frame}{}
  \titlepage
\end{frame}

\section{Bevezetés}

\begin{frame}{A feladat és a probléma}

\bigskip

A feladat: a szövegben felismerni és előre definiált osztályok valamelyikébe besorolni a neveket 

\bigskip

A probléma:
\begin{itemize}
\item a sztenderd adathalmazok korlátozott méretűek és témaspecifikusak 
\item robusztus rendszerek építéséhez nagyméretű és heterogén tulajdonnév-annotált korpuszokra van szükség
\item a kézi annotálás rendkívül költséges és erőforrás-igényes
\end{itemize}

\end{frame}

\begin{frame}{A kihívás}

az annotációs költség csökkentése \nyil automatikusan annotált korpuszok előállítása

\begin{itemize}
\item meglévő korpuszok egyesítése \nyil az annotációs sémák és címkekészletek különbözősége 
\item rendelkezésre álló korpuszon tanított tagger futtatása új szövegen \nyil a domének közötti különbség teljesítménycsökkenést okoz
\item közösségi tartalmak felhasználása \nyil leképezés a sztenderd címkekészletekre
\end{itemize}

\smallskip

\begin{nix}
A fenti megközelítések kombinációja: \\ automatikus eszközökkel tulajdonnév-annotált korpuszok építése a Wikipédiából
\end{nix}

\end{frame}

\section{Wikipédia és tulajdonnév-felismerés}

\begin{frame}{Wikipédia és NER}

\bigskip

\begin{nix}
mivel a Wikipédia címszavainak nagy része tulajdonnév, kiválóan használható tulajdonnév-felismeréshez
\end{nix}

\bigskip

Alkalmazási módjai:
\smallskip
\begin{itemize}
\item nagyméretű névlisták előállítása 
\item a Wikipédia kategóriacímkéinek kinyerése és beépítése NER rendszerekbe
\item tanító korpusz előállítása a Wikipédia szövegbázisából
\begin{itemize}
\item az inherens kategóriastruktúra felhasználásával
\item a szócikkek első mondatának felhasználásával
\end{itemize} 
\end{itemize}

\end{frame}

\begin{frame}{Our Approach}

\begin{nix}
Mapping the DBpedia ontology classes to standard NE tags, and assign them to WP entities
\end{nix}

\bigskip

Our method is largely language-independent \nyil sufficiently large corpora for less resourced languages can be built 

\bigskip

\begin{nix}
This is the first automatically annotated corpus for Hungarian 
\end{nix}

\end{frame}

\section{Korpuszépítés}

\subsection*{A módszer}

% TODO: bigskip?
\begin{frame}{Korpuszépítés}
  \begin{itemize}
  \vitem A Wikipédia két feltételezett tulajdonsága:
    \begin{enumerate}
    \vitem a szócikkek nagy része entitásokról szól
    \vitem a kereszthivatkozások azonosítják a tulajdonneveket a szövegben
    \end{enumerate}
  \vitem Módszerünk hasonló Nothman et al. (2008)-hoz:
    \begin{enumerate}
    \vitem a Wikipédia-cikkeket entitásosztályokba soroljuk;
    \vitem a cikkeket mondatokra bontjuk;
    \vitem felcímkézzük a tulajdonneveket a szövegben;
    \vitem kiszűrjük a rossz minőségű mondatokat.
    \end{enumerate}
  \end{itemize}
\end{frame}

\subsection{Cikkek, mint entitások}
\begin{frame}{Cikkek, mint entitások}
  \bigskip
  Cikkek entitásosztályokba sorolása: % TODO what?
  \smallskip
  \begin{itemize}
  \vitem Szokásos megoldás: felügyelt tanulás (infoboxok, első mondat)
  \vitem A mi megközelítésünk: entitás lista a DBpedia alapján
    \begin{itemize}
    \vitem OWL ontológia kb. 320 osztállyal: kézileg leképeztük CoNLL (Szeged) kategóriákra
    \vitem Minden entitást a legspecifikusabb osztály szerinti kategóriába raktunk
    \end{itemize}
  \item Csökkent fedés, jobb pontosság
  \end{itemize}
\end{frame}

\subsection{Tulajdonnevek címkézése}
\begin{frame}{Tulajdonnevek címkézése}
  \bigskip
  \begin{itemize}
  \vitem Tulajdonnevek felismerése szövegben:
    \begin{itemize}
    \vitem Kereszthivatkozások, ha a hivatkozott cikk szerepel az entitáslistában
    \vitem Az aktuális cikk címe, ha szerepel a listában
    \end{itemize}
  \vitem Címkézés: a DBpedia--Szeged leképezés alapján
  \vitem Kivételek:
    \begin{itemize}
    \vitem Cikkek, amik nem entitások: nagybetűvel kell kezdődjön a szócikk
    \vitem Jelöletlen entitások: megpróbáljuk őket korábbi tulajdonnevekhez kötni
    \vitem Speciális esetek: tulajdonnév-köznév összetételek, mondatkezdő főnevek, stb.
    \end{itemize}
  \end{itemize}
\end{frame}

\subsection{Mondatok szűrése}
\begin{frame}{Mondatok szűrése}
  \bigskip
  Csak a teljesen felcímkézett mondatokat tartjuk meg. Eldobjuk:
  \smallskip
  \begin{itemize}
  \vitem azokat a mondatokat, amiben címkézetlen hivatkozások vagy nagybetűs szavak vannak
  \vitem a mondattöredékeket:
    \begin{itemize}
    \vitem amikben nincs mondatvégi írásjel
    \vitem állítmány nélküli mondatok %TODO: tényleg?
    \end{itemize}
  \end{itemize}
\end{frame}

\section{Problémás esetek, hibaelemzés}

%\begin{frame}{The Hungarian Corpus}
%  The algorithm is, in theory, language-agnostic. We tested this property by
%  applying it to a medium-sized language, Hungarian. \\
%  Difference from bigger, well-resourced languages:
%  \begin{itemize}
%  \vitem E % TODO: ide mit?
%  \end{itemize}
%\end{frame}

%TODO: the "Hungarian Corpus" phrase must appear in the frame title 
\begin{frame}{Differences from English}
  \bigskip
  \begin{itemize}
  \vitem Language processing tools
    \begin{itemize}
    \vitem Inflectional morphology: different set of tools were needed 
    \vitem We developed a flexible framework that enables the plugging of any NLP tool
    % TODO: what kind of linguistic tools are needed: table?
    \end{itemize}
  \vitem Wikipedia size and coverage
    \begin{itemize}
    \vitem Hungarian is medium-size language: \nagytilde 200k articles vs.~\nagytilde 4M English pages
    \vitem Cross-references may point to English pages
    \vitem Templates are in Hungarian: per-language template configuration
    \end{itemize}
  \end{itemize}
\end{frame}

\begin{frame}{Differences from English -- cont.}
  \bigskip
  \begin{itemize}
  \vitem DBpedia coverage
    \smallskip
    \begin{itemize}
    \vitem Internationalized chapters for big languages only
    \vitem "Hungarian" DBpedia list contains English articles % Nem egyértelmű
      \begin{itemize}
      \vitem Hungarian equivalents must be added to the list % Added: lásd előző oldal 2/2
      \vitem Hungarian entities without an English page are missing:
             the NER language model might not be accurate
      \vitem Future work: extend the list with the missing Hungarian entities
      \end{itemize}
    \end{itemize}
  \end{itemize}
\end{frame}

\begin{frame}{Differences from English -- cont.}
  \bigskip
  \begin{itemize}
  \vitem NE conventions
    \smallskip
    \begin{itemize}
    \vitem Szeged NER guideline: similar to CoNLL
    \vitem Common nouns in Hungarian: names of languages, religions, months, etc. \nyil more consistent MISC category
    \vitem NEs can be subject of compounding \nyil the joint common noun can modify the original sense and category of the NE:
    \end{itemize}
  \end{itemize}
  \bigskip
  \begin{Examplee}
  \small{WorldCom {\normalfont(\texttt{ORG})} \nyil WorldCom-botrány ['WorldCom scandal'] {\normalfont(\texttt{O})}}
  \end{Examplee}
\end{frame}

\section{A korpuszok leírása}

\begin{frame}{Data Format}

\begin{tabular}{llllll}
\textbf{token} & \textbf{text/link} & \textbf{WP link} & \textbf{POS} & \textbf{lemma} & \textbf{NE} \\
This & text & 0 &    DT &   This & O \\
is &   text & 0 &    VBZ &  be &   O \\
a &    text & 0 &    DT &   a  &    O\\
list & text & 0 &    NN &   list  & O\\
of &   text & 0 &    IN &   of &     O\\
films  & B-link & films &  NNS &  film & O\\
by &   text & 0 &    IN &   by &     O\\
year & text & 0 &    NN &   year & O\\
produced & text & 0 &    VBN &  produce & O\\
in &   text & 0 &    IN &   in &     O\\
the &  text & 0 &    DT &   the &   O\\
country &text & 0 &    NN &   country &O\\
of &   text & 0 &    IN &   of &     O\\
South &  B-link & South Korea &  NNP &  South  &B-LOC\\
Korea &  I-link & South Korea &  NNP &  Korea   & I-LOC\\
\end{tabular}

\end{frame}

\begin{frame}{A korpusz elérhetősége}
\bigskip
A korpuszok
\begin{itemize}
\vitem a CC-BY-SA 3.0 Unported licensz alatt kerültek publikálásra
\vitem szabadon letölthetőek a \url{http://hlt.sztaki.hu/resources/} honlapról
\vitem szintén elérhetőek a META-SHARE-en keresztül. Utóbbi
  \begin{itemize}
  \vitem egy nyílt rendszer, mely lehetővé teszi a nyelvi erőforrások megosztását
  \vitem létrehozója a META-NET, az Európai Bizottság által alapított nyelvtechnológiai hálózat
  \vitem \url{http://www.meta-net.eu/meta-share}
  \end{itemize}
\end{itemize}

\end{frame}

\section{Kiértékelés}

\begin{frame}{Silver Standard Corpora}

\bigskip

Silver standard corpora can be useful for improving NER in several ways:

\smallskip

\begin{itemize}
\vitem for less resourced languages they can serve as training corpora in lieu of gold standard datasets
\vitem they can serve as training sets for domains differing from newswire
\vitem they can be sources of huge entity lists
\vitem they can be sources of feature extraction
\end{itemize}

\end{frame}

\begin{frame}{Evaluation}

\bigskip

A maxent NE tagger was used with the feature set consisting of

\smallskip

\begin{itemize}
\vitem gazetteer features
\vitem sentence start and end position
\vitem Boolean-valued orthographic properties of the word form
\vitem string-valued properties of the word form
\vitem morphological information
\end{itemize}

\bigskip

The CoNLL standard method was used for evaluation.

\end{frame}

\begin{frame}{Corpus Size and NE Density}

English: 

\begin{center}
\begin{tabular}{llll}
\hline  & \bf enwiki & \bf enwiki filtered & \bf CoNLL \\ \hline
token & 60,520,819 & 21,718,854 & 302,811 \\
NE & 3,169,863 & 3,169,863 & 50,758 \\
NE density & 5.23\% & 14.59\% & 16.76\% \\ \hline
\end{tabular}
\end{center}

%\smallskip

Hungarian:

\begin{center}
\begin{tabular}{llll}
\hline  & \bf huwiki & \bf huwiki filtered  & \bf Szeged NER \\ \hline
token & 19,108,027 & 3,512,249  & 225,963 \\
NE & 456,281 & 456,281  & 25,896 \\
NE density & 2.38\% & 12.99\%  & 11.46\% \\ \hline 
\end{tabular}
\end{center}

Quite low NE density \nyil sentences without NEs were filtered out from the evaluation datasets

\end{frame}

\begin{frame}{Experiments}

\begin{nix}
Our experiments were performed with a sample of 3.5 million tokens
\end{nix}

\bigskip

\begin{itemize}
\item \textit{CoNLL} train -- \textit{CoNLL} test
\item \textit{enwiki} train -- \textit{enwiki} test
\item \textit{enwiki} train -- \textit{CoNLL} test
\item \textit{CoNLL} train with \textit{wikilists} (gazetteer lists extracted from WP) -- \textit{CoNLL} test
\item \textit{CoNLL} train with \textit{wikitags} (labeling the CoNLL datasets by the model trained on WP, and giving these labels as extra features) -- \textit{CoNLL} test
\end{itemize}

\end{frame}

\begin{frame}{English Results}

\begin{center}
\begin{tabular}{lllll}
\hline \bf Train & \bf Test & \bf P (\%) & \bf R (\%) & \bf F (\%) \\ \hline
CoNLL & CoNLL & 85.13 & 85.13 & 85.13 \\
enwiki & enwiki & 72.46 & 73.33 &  72.89 \\
enwiki & CoNLL & 56.55 & 49.77 & 52.94 \\
CoNLL with wikilists & CoNLL & 86.33 & 86.35 & 86.34 \\
CoNLL with wikitags & CoNLL & 85.88 & 85.94 & 85.91 \\
\hline
\end{tabular}
\end{center}

\begin{itemize}
\item training and testing across different corpora decreases F-measure
\item the NE tagger trained on WP does not achieve as high performance tested against CoNLL test set as one trained on its own train set
\item gazetteer lists extracted from WP and training with extra features given by the model trained on WP improve F-measure
\end{itemize}

\end{frame}

\begin{frame}{Hungarian Results}

\begin{center}
\begin{tabular}{lllll}
\hline \bf Train & \bf Test & \bf P (\%) & \bf R (\%) & \bf F (\%) \\ \hline
Szeged  & Szeged  & 94.50 & 94.35 & 94.43 \\
huwiki & huwiki & 90.64 & 88.91 &  \textbf{89.76} \\
huwiki & Szeged  & 63.08 & 70.46 & 66.57 \\
Szeged  with wikilists & Szeged  & 95.48 & 95.48 & 95.48 \\
Szeged  with wikitags & Szeged  & 95.38 & 94.92 & 95.15 \\
\hline
\end{tabular}
\end{center}

\begin{itemize}
\vitem F-measures for Hungarian are higher 
\begin{itemize}
\vitem the MISC category is more consistent
\vitem the Szeged NER corpus is highly accurately tagged 
\end{itemize}
\vitem the Hungarian corpus can serve as a training corpus to build NE taggers for non-newswire domains
\end{itemize}

\end{frame}

%\begin{frame}{Results -- analysis}

%\begin{itemize}
%\vitem training and testing across different corpora decreases F-measure
%\vitem the NE tagger trained on WP does not achieve as high performance tested against CoNLL test set as one trained on its own train set
%\vitem gazetteer lists extracted from WP and training with extra features given by the model trained on WP improve F-measure
%\vitem F-measures for Hungarian are higher 
%\begin{itemize}
%\vitem the MISC category is more consistent
%\vitem the Szeged NER corpus is highly accurately tagged 
%\end{itemize}
%\end{itemize}
%  \vfill

%\end{frame}

\section{Összegzés}
\begin{frame}{Conclusions}
  \begin{itemize}
  \vitem We presented freely available NE tagged corpora for English and Hungarian, fully automatically generated from WP
  \vitem We applied a new approach: mapping DBpedia ontology classes to standard CoNLL NE tags, and assigning them to WP entities
  \vitem The huge amount of WP articles opens the possibility of building large enough corpora for less resourced languages such as Hungarian
  \vitem Silver standard corpora can improve NER accuracy in more ways
  \vitem The Hungarian corpus can serve as a training corpus to build NE taggers for non-newswire domains
  \end{itemize}
  \vfill
\end{frame}

\begin{frame}

\bigskip

\bigskip

{\huge Köszönjük a figyelmet!}

\bigskip
\bigskip
\bigskip

e-mail: \\ {\tt nemeskey.david@sztaki.mta.hu \\ simon.eszter@nytud.mta.hu} \\

\bigskip
\bigskip

URL: \\ {\tt http://hlt.sztaki.hu/resources/}


\end{frame}

\end{document}


