%
% File acl2012.tex
%
% Contact: Maggie Li (cswjli@comp.polyu.edu.hk), Michael White (mwhite@ling.osu.edu)
%%
%% Based on the style files for ACL2008 by Joakim Nivre and Noah Smith
%% and that of ACL2010 by Jing-Shin Chang and Philipp Koehn


\documentclass[11pt]{article}
\usepackage{acl2012}
\usepackage{times}
\usepackage{latexsym}
\usepackage{amsmath}
\usepackage{multirow}
\usepackage{url}
\usepackage[utf8]{inputenc}
\DeclareMathOperator*{\argmax}{arg\,max}
\setlength\titlebox{6.5cm}    % Expanding the titlebox

\title{XXX}

%\author{Dávid Márk Nemeskey \\
%  Research Institute for Computer Science and Automation \\
%  Hungarian Academy of Sciences \\
%  H-1111 Lágymányosi út 11, Budapest \\
%  {\tt nemeskey.david@sztaki.hu} \\\And
%  Eszter Simon \\
%  Research Institute for Linguistics \\
%  Hungarian Academy of Sciences \\
%  H-1068 Benczúr utca 33, Budapest \\
%  {\tt eszter@nytud.hu} \\}

\date{}

\begin{document}
\maketitle
\begin{abstract}
  This document contains the instructions for preparing a camera-ready manuscript for the proceedings of ACL2012. The document itself conforms to its own specifications, and is therefore an example of what your manuscript should look like. These instructions should be used for both papers submitted for review and for final versions of accepted papers. Authors are asked to conform to all the directions reported in this document.
\end{abstract}

\section{Introduction}

One of the main trends in NLP is the need for reducing the annotation cost. After examining the calls for papers of nowadays conferences and workshops there seems to be three major directions to achieve this aim. 

The first approach is the application of unsupervised and semi-supervised learning techniques. Researchers following this approach try to not only reduce, but avoid the annotation cost with minimizing the need for annotated data. These methods can be applied to any language or genre for which adequate raw text resources are available. 

The second approach is using collaborative annotation and/or collaboratively constructed resources, such as Wikipedia, Wiktionary or Linked Open Data. Researchers apply the Web-as-corpus approach, often dealing with such datasets as the Google 5G 1T collection or DBPedia. Thanks for the huge amount of Wikipedia articles, the possibility of building large enough corpora is given for less resourced languages, as well. 

The third approach to reduce the manual intervention required in language resource production, and thus ultimately production costs is automatically generating resources or merging existing ones. However merging language resources faces NLP researchers with the problem of combining different annotation schemas. 

Named Entity Recognition (NER) is one of the important subtasks of Information Extraction. NER contains two main substeps: recognizing the Named Entities (NEs) in unstructured text and classifying them into preselected classes

mostly newspaper articles, domain-specific

In this article we present a Named Entity tagged corpus for English -- fully automatically generated from Wikipedia. 
The applied annotation schema is the CoNLL tagset, which is one of the major standards applied in the field of Named Entity Recognition (NER). 

The paper is structured as follows. 

\section{Related Work}
% Wikify! 
% Ide kell? Vagy Conclusion elé?

\section{Creation of the Corpus}  % Dzsííízösz

% Introduction: Wikipedia + DBpedia
The corpora are available in the tab separated format used by ConLL (XXX, CITE). In this format, one line contains one word, its linguistic features and the named entity tags. Sentence boundaries are marked by empty lines. The linguistic features include the lemmatized form of the word, part-of-speech tag and chunk tag. Two named entity tags are included with each word: one is the most specific DBpedia category it belongs to, if any; the other is the ConLL named entity tag. While the named entity tags can be considered "gold", the linguistic features are provided on a "best-effort" basis (XXX: check the LREC paper to find a better wording for this).
XXX: past or present tense?

We decided to use Wikipedia because

The categories of Wikipedia are too numerous and ; not a tree;

Wikipedia proves to be a rich source of information about its entities; aside from the article text, a huge amount of data is encompassed in infoboxes, templates and the category structure. Since our goal is to create a NER training corpus, our interest is limited to the article text. Templates are removed from the articles; however, the names of the templates have been retained for XXX, but all data (including infobox) 

Beside regular articles, Wikipedia also contains redirect and disambiguation pages, which require special treatment. Redirect pages are empty and are safe to ignore altogether. Disambiguation pages contain text, usually in the form of an enumeration of page links with a short description. Full sentences are scarce and the structure of these pages makes it hard to extract valid training sentences from them. Therefore, disambiguation pages have been dropped from the corpus as well.

The corpus is based on the Wikipedia snapshot as of XXX. The xml files were parsed with the XXX (Attila) parser. The raw text was tokenized with a modified version of the XXX sentence tokenizer in NLTK (CITE). The words were lemmatized with XXX and POS tagged with HunPos (CITE). Chunking was performed with HunChunk (CITE). Experiments have shown that HunPos obtains STAT f-score on the Brown corpus; HunChunk performs at around STAT, CITE?.

\subsection{Extraction of the Training Sentences}

A NER training corpus consists of sentences that accurately tag named entities occurring in them. In order to automatically prepare such sentences, two tasks need to be performed: identification of entities in the sentence and tagging them with the correct tag. Sentences, for which accurate tagging could not be accomplished, must be removed from the corpus. In this section, we describe how these steps were implemented, as well as the post-filtering we employed to dispose of correctly tagged but unusable training sentences.

% TODO: we want to be on the strict side!

\subsubsection{Entity identification}

Our approach to entity identification relies on Wikipedia cross-references found in the article text. We assume that individual Wikipedia articles describe named entities. A link to an article can then be perceived as a mapping that identifies its anchor text with a particular entity. There are exceptions to this rule, however. Disambiguation pages are used when the page title is ambiguous, and can refer to several entities. If a link points to a disambiguation page, we acknowledge the anchor text as an entity, but mark it unknown. A similar measure is taken if the link target XXX does not exist. Links to redirect pages are resolved to point instead to the redirect target, after which they are handled as regular cross-references.

Not all entities are marked with links, however. Authors may neglect to add XXX references to their text. Furthermore, in a Wikipedia article XXX, typically only the first mention of a particular entity is linked to the corresponding page. Subsequent mentions are unmarked and often incomplete -- e.g. family names are used instead of full names.

% Numbers, years.

\subsubsection{Tagging}

We use DBpedia as 

Only sentences for whose entities the correct category could be obtained from DBpedia are added to the corpus; those that contain unknown entities are discarded. Sentences without any entities are retained as well, so that the corpus mirrors (XXX the distribution of the entities in the corpus mirrors their distribution in Wikipedia?) the distribution of entities in Wikipedia as closely as possible; this is also in line with how the ConLL corpus is assembled. (XXX or is it)

Unknown entities might or might not be 

% TODO: monetary, etc. units?

% 3. Training sentence extraction
%   a. only known links
%   b. title parts
%   c. parts of links mention earlier
%   d. date regexp
% recall+precision?

\subsubsection{Filtering}

Some sentences that were selected by the process above

% 4. Test sentence extraction?
%   a. unknown links, no unlinked names
%   b. might be good for quality evaluation / recursive tagging (next paper?)
% 5. teaching the original DBpedia categories?

\subsection{The DBpedia -- ConLL mapping}

% Introduction + entity -> class -> category.

The classes in DBpedia are organized into a type hierarchy, available as an OWL (CITE) ontology. The ontology contains the 320 most frequent categories of Wikipedia, arranged into a taxonomy under the base class \texttt{owl:Thing}. Most of the classes belong into the 5 largest sub-hierarchies (here named after their topmost classes XXX): \texttt{Agent} (which is further divided into \texttt{Person} and \texttt{Organisation}), \texttt{Event}, \texttt{Place}, \texttt{Species} and \texttt{Work}. The taxonomy is rather flat: the top level contains 44 classes and there are several nodes with a branching factor of 20.

It is not difficult to see the parallels XXX between the DBpedia sub-hierarchies \texttt{Person}, \texttt{Organisation} and \texttt{Place} and the ConLL NER categories \texttt{PER}, \texttt{ORG} and \texttt{LOC}. The fourth category, MISC is more elusive; according to the ConLL NER annotation guide (CITE?: http://www.cnts.ua.ac.be/conll2003/ner/annotation.txt), the sub-hierarchies \texttt{Event} and \texttt{Work} belong to this category, as well as various other class outside the main sub-hierarchies. 

While the correspondence described above holds for most classes in the XXX sub-hierarchies, there are some exceptions. For instance, the class \texttt{SportsLeague} is part of the \texttt{Organisation} sub-hierarchy, but in the ConLL corpus, sport leagues are tagged as MISC. Similarly, the class \texttt{SportsTeamSeason} is under \texttt{Event}, by which account it would be classified as MISC; however, ConLL does not recognize such entities. To account for these misclassification, as well as the miscellaneous classes outside the main hierarchies, we have introduced a mapping mechanism. We created a file of DBpedia class -- NER category mappings. Whenever an entity is evaluated, we look up its class, as well as the ancestors of its class, in the mapping, and assign to the entity the category of the class that matches it most closely -- i.e. the category of the entity's class, if it is in the list, or that of its closest ancestor. XXX!!! If no match is found, the entity is tagged with \texttt{0}. Since we take advantage of the inheritance hierarchy, the mapping list remains short: it contains only the root classes of the main sub-hierarchies, exceptions of the like XXX mentioned above and the various classes that belong to the MISC category according to the ConLL annotation guideline.

As of version 3.7, the DBpedia ontology allows multiple superclasses, making the hierarchy a directed acyclic graph. This introduces the problem of selecting the right superclass, and hence, ConLL mapping for classes with more than one parent. In version 3.7, the only such class is \textit{Library}, which can be traced back to both \textit{Place} and \textit{Organisation}. There are two ways to tackle this problem. The first is to follow the DBpedia guidelines and only retain the first parent. The second method is manually deciding the right NER mapping for the ambiguous XXX class and adding it to the mapping file. The first approach, unfortunately, already fails to assign the correct ConLL NER tag to our only example. The reason behind this failure is that multiple parents in an ontology show an inherent ambiguity in the class in question: it is entirely natural to think of a library as a location as well as a public organization. DBpedia's guidelines prefers Organisation; ConLL's LOC. Therefore, only the second, manual approach ensures correct tagging. XXX

The full mapping can be found in LINK Appendix A (XXX or not).

\subsection{Remarks}

Strictly speaking, our original assumption of equating Wikipedia articles with named entities is not valid: many pages describe common nouns (Book, Aircraft), currencies (Euro, Dollar), calendrical units XXX (January, the year 1999) or other concepts that fall outside the scope of NER. However, such entities are either missing from DBpedia altogether, or belong to classes mapped to the NER category \texttt{0} (not an entity). This "ontological filtering" ensures that 
% TODO: this is not true, we need filters!

In this paper, we describe % not only ConLL

\section{Creation of the Hungarian Corpus}

% Good for middle-sized languages
% Inconsistent template names, esp. across languages
% Comparison of the Hungarian DBpedia vs English + mapping

\section{Evaluation}

% vs. ConLL / Szeged
% Corpus sizes, performance, tagging conventions

\section{Conclusions and Future Work}

\section{General Instructions}

Manuscripts must be in two-column format. Exceptions to the two-column format include the title, 
authors' names and complete addresses, which must be centered at the top of the first page, 
and any full-width figures or tables (see the guidelines in Subsection~\ref{ssec:first}). {\bf Type single-spaced}. 
Start all pages directly under the top margin. See the guide-lines later regarding formatting the first page. Do not number the pages.

\subsection{Electronically-available resources}

ACL2012 provides this description in \LaTeX2e (acl2012.tex) and PDF format (acl2012.pdf), along with the LATEX2e style file used to format it (acl2012.sty) and an ACL bibliography style (acl.bst). These files are all available at \url{http://www.acl2012.org}.  A Microsoft Word template file (acl2012.dot) is also available at the same URL. We strongly recommend the use of these style files, which have been appropriately tailored for the ACL2012 proceedings. If you have an option, we recommend that you use the \LaTeX2e version. \textbf{If you will be using the Microsoft Word template, we suggest that you anonymize your source file so that the pdf produced does not retain your identity.} This can be done by removing any personal information from your source
document properties.


\subsection{Format of Electronic Manuscript}
\label{sect:pdf}

For the production of the electronic manuscript you must use Adobe's
Portable Document Format (PDF). This format can be generated from
postscript files: on Linux/Unix systems, you can use {\tt ps2pdf} for this
purpose; under Microsoft Windows, you can use Adobe's Distiller, or
if you have {\tt cygwin} installed, you can use {\tt dvipdf} or
{\tt ps2pdf}.  Note
that some word processing programs generate PDF which may not include
all the necessary fonts (esp. tree diagrams, symbols). When you print
or create the PDF file, there is usually an option in your printer
setup to include none, all or just non-standard fonts.  Please make
sure that you select the option of including ALL the fonts.  {\em Before sending it, test your PDF by printing it from a computer different from the one where it was created}. Moreover,
some word processor may generate very large postscript/PDF files,
where each page is rendered as an image. Such images may reproduce
poorly.  In this case, try alternative ways to obtain the postscript
and/or PDF.  One way on some systems is to install a driver for a
postscript printer, send your document to the printer specifying
``Output to a file'', then convert the file to PDF.

Additionally, it is of utmost importance to specify the {\bf US-Letter format} (8.5in $\times$ 11in) when formatting the paper. When working with {\tt dvips}, for instance, one should specify {\tt -t letter}.

Print-outs of the PDF file on US-Letter paper should be identical to the
hardcopy version.  If you cannot meet the above requirements about the
production of your electronic submission, please contact the
publication chair above as soon as possible.


\subsection{Layout}
\label{ssec:layout}

Format manuscripts two columns to a page, in the manner these
instructions are formatted. The exact dimensions for a page on US-letter
paper are:

\begin{itemize}
\item Left and right margins: 1in
\item Top margin:1in
\item Bottom margin: 1in
\item Column width: 3.15in
\item Column height: 9in
\item Gap between columns: 0.2in
\end{itemize}

\noindent Papers should not be submitted on any other paper size. If you cannot meet the above requirements about the production of your electronic submission, please contact the publication chair above as soon as possible.

\subsection{Fonts}

For reasons of uniformity, Adobe's {\bf Times Roman} font should be
used. In \LaTeX2e{} this is accomplished by putting

\begin{quote}
\begin{verbatim}
\usepackage{times}
\usepackage{latexsym}
\end{verbatim}
\end{quote}
in the preamble. If Times Roman is unavailable, use {\bf Computer
  Modern Roman} (\LaTeX2e{}'s default).  Note that the latter is about
  10\% less dense than Adobe's Times Roman font.


\begin{table}[h]
\begin{center}
\begin{tabular}{|l|rl|}
\hline \bf Type of Text & \bf Font Size & \bf Style \\ \hline
paper title & 15 pt & bold \\
author names & 12 pt & bold \\
author affiliation & 12 pt & \\
the word ``Abstract'' & 12 pt & bold \\
section titles & 12 pt & bold \\
document text & 11 pt  &\\
captions & 11 pt & \\
abstract text & 10 pt & \\
bibliography & 10 pt & \\
footnotes & 9 pt & \\
\hline
\end{tabular}
\end{center}
\caption{\label{font-table} Font guide. }
\end{table}

\subsection{The First Page}
\label{ssec:first}

Center the title, author's name(s) and affiliation(s) across both
columns. Do not use footnotes for affiliations.  Do not include the
paper ID number assigned during the submission process.
Use the two-column format only when you begin the abstract.

{\bf Title}: Place the title centered at the top of the first page, in
a 15 point bold font.  (For a complete guide to font sizes and styles, see Table~\ref{font-table}.)
Long title should be typed on two lines without
a blank line intervening. Approximately, put the title at 1in from the
top of the page, followed by a blank line, then the author's names(s),
and the affiliation on the following line.  Do not use only initials
for given names (middle initials are allowed). Do not format surnames
in all capitals (e.g., ``Zhou,'' not ``ZHOU'').  The affiliation should
contain the author's complete address, and if possible an electronic
mail address. Leave about 0.75in between the affiliation and the body
of the first page. The title, author names and addresses should be completely identical to those entered to the electronic paper submission website in order to maintain the consistency of author information among all publications of the conference.

{\bf Abstract}: Type the abstract at the beginning of the first
column.  The width of the abstract text should be smaller than the
width of the columns for the text in the body of the paper by about
0.25in on each side.  Center the word {\bf Abstract} in a 12 point
bold font above the body of the abstract. The abstract should be a
concise summary of the general thesis and conclusions of the paper.
It should be no longer than 200 words. The abstract text should be in 10 point font.

{\bf Text}: Begin typing the main body of the text immediately after
the abstract, observing the two-column format as shown in
the present document. Do not include page numbers.

{\bf Indent} when starting a new paragraph. For reasons of uniformity,
use Adobe's {\bf Times Roman} fonts, with 11 points for text and
subsection headings, 12 points for section headings and 15 points for
the title.  If Times Roman is unavailable, use {\bf Computer Modern
  Roman} (\LaTeX2e's default; see section \ref{sect:pdf} above).
Note that the latter is about 10\% less dense than Adobe's Times Roman
font.

\subsection{Sections}

{\bf Headings}: Type and label section and subsection headings in the
style shown on the present document.  Use numbered sections (Arabic
numerals) in order to facilitate cross references. Number subsections
with the section number and the subsection number separated by a dot,
in Arabic numerals. Do not number subsubsections.

{\bf Citations}: Citations within the text appear
in parentheses as~\cite{Gusfield:97} or, if the author's name appears in
the text itself, as Gusfield~\shortcite{Gusfield:97}. Append lowercase letters to the year in cases of ambiguities. Treat double authors as in~\cite{Aho:72}, but write as in~\cite{Chandra:81} when more than two authors are involved. Collapse multiple citations as in~\cite{Gusfield:97,Aho:72}. Also refrain from using full citations as sentence constituents. We suggest that instead of
\begin{quote}
  ``\cite{Gusfield:97} showed that ...''
\end{quote}
you use
\begin{quote}
``Gusfield \shortcite{Gusfield:97}   showed that ...''
\end{quote}

If you are using the provided \LaTeX{} and Bib\TeX{} style files, you
can use the command \verb|\newcite| to get ``author (year)'' citations.

As reviewing will be double-blind, the submitted version of the papers should not include the
authors' names and affiliations. Furthermore, self-references that
reveal the author's identity, e.g.,
\begin{quote}
``We previously showed \cite{Gusfield:97} ...''
\end{quote}
should be avoided. Instead, use citations such as
\begin{quote}
``Gusfield \shortcite{Gusfield:97}
previously showed ... ''
\end{quote}

Please do not  use anonymous
citations and  do not include acknowledgements when submitting your papers. Papers that do not conform
to these requirements may be rejected without review.

\textbf{References}: Gather the full set of references together under
the heading {\bf References}; place the section before any Appendices,
unless they contain references. Arrange the references alphabetically
by first author, rather than by order of occurrence in the text.
Provide as complete a citation as possible, using a consistent format,
such as the one for {\em Computational Linguistics\/} or the one in the
{\em Publication Manual of the American
Psychological Association\/}~\cite{APA:83}.  Use of full names for
authors rather than initials is preferred.  A list of abbreviations
for common computer science journals can be found in the ACM
{\em Computing Reviews\/}~\cite{ACM:83}.

The \LaTeX{} and Bib\TeX{} style files provided roughly fit the
American Psychological Association format, allowing regular citations,
short citations and multiple citations as described above.

{\bf Appendices}: Appendices, if any, directly follow the text and the
references (but see above).  Letter them in sequence and provide an
informative title: {\bf Appendix A. Title of Appendix}.

\textbf{Acknowledgment} sections should go as a last section immediately
before the references. Do not number the acknowledgement section.

\subsection{Footnotes}

{\bf Footnotes}: Put footnotes at the bottom of the page. They may
be numbered or referred to by asterisks or other
symbols.\footnote{This is how a footnote should appear.} Footnotes
should be separated from the text by a line.\footnote{Note the
line separating the footnotes from the text.}  Footnotes should be in 9 point font.

\subsection{Graphics}

{\bf Illustrations}: Place figures, tables, and photographs in the
paper near where they are first discussed, rather than at the end, if
possible.  Wide illustrations may run across both columns and should be placed at
the top of a page. Color illustrations are discouraged, unless you have verified that
they will be understandable when printed in black ink.

{\bf Captions}: Provide a caption for every illustration; number each one
sequentially in the form:  ``Figure 1. Caption of the Figure.'' ``Table 1.
Caption of the Table.''  Type the captions of the figures and
tables below the body, using 10 point text.

\section{Translation of non-English Terms}

It is also advised to supplement non-English characters and terms
with appropriate transliterations and/or translations
since not all readers understand all such characters and terms.

Inline transliteration or translation can be represented in
the order of: original-form transliteration ``translation''.

\section{Length of Submission}
\label{sec:length}

Long papers may consist of up to eight (8) pages of content and an
additional two (2) pages of references, and short papers may consist
of up to four (4) pages of content and two (2) additional pages of
references.  Papers that do not conform to the specified length and
formatting requirements are subject to re-submission.

\section{Other Issues}
 
Those papers that had software and/or dataset submitted for the review process, should also submit it 
with the camera-ready paper. Besides, the software and/or dataset should not be anonymous. 

Please note that the publications of ACL2012 will be publicly available at ACL Anthology 
(http://aclweb.org/anthology-new/) on July 2nd, 2012, one week before the start of the conference. 
Since some of the authors may have plans to file patents related to their papers in the conference, 
we are sending this reminder that July 2nd, 2012 may be considered to be the official publication date, 
instead of the opening day of the conference.

%\section*{Acknowledgments}

%The authors are grateful to Attila Zséder and Gábor Recski for their respective work on Wikipedia parsing and {\tt hunner}.
%Do not number the acknowledgment section. Do not include this section when submitting your paper for review.

\begin{thebibliography}{}

\bibitem[\protect\citename{Aho and Ullman}1972]{Aho:72}
Alfred~V. Aho and Jeffrey~D. Ullman.
\newblock 1972.
\newblock {\em The Theory of Parsing, Translation and Compiling}, volume~1.
\newblock Prentice-{Hall}, Englewood Cliffs, NJ.

\bibitem[\protect\citename{{American Psychological Association}}1983]{APA:83}
{American Psychological Association}.
\newblock 1983.
\newblock {\em Publications Manual}.
\newblock American Psychological Association, Washington, DC.

\bibitem[\protect\citename{{Association for Computing Machinery}}1983]{ACM:83}
{Association for Computing Machinery}.
\newblock 1983.
\newblock {\em Computing Reviews}, 24(11):503--512.

\bibitem[\protect\citename{Chandra \bgroup et al.\egroup }1981]{Chandra:81}
Ashok~K. Chandra, Dexter~C. Kozen, and Larry~J. Stockmeyer.
\newblock 1981.
\newblock Alternation.
\newblock {\em Journal of the Association for Computing Machinery},
  28(1):114--133.

\bibitem[\protect\citename{Gusfield}1997]{Gusfield:97}
Dan Gusfield.
\newblock 1997.
\newblock {\em Algorithms on Strings, Trees and Sequences}.
\newblock Cambridge University Press, Cambridge, UK.

\end{thebibliography}

\end{document}
