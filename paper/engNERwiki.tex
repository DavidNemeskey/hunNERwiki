%
% File acl2012.tex
%
% Contact: Maggie Li (cswjli@comp.polyu.edu.hk), Michael White (mwhite@ling.osu.edu)
%%
%% Based on the style files for ACL2008 by Joakim Nivre and Noah Smith
%% and that of ACL2010 by Jing-Shin Chang and Philipp Koehn

\documentclass[11pt]{article}
\usepackage{acl2012}
\usepackage{times}
\usepackage{latexsym}
\usepackage{amsmath}
\usepackage{multirow}
\usepackage{url}
\usepackage[utf8]{inputenc}
\DeclareMathOperator*{\argmax}{arg\,max}
\setlength\titlebox{6.5cm}    % Expanding the titlebox

\title{Automatically generated NE tagged corpora for English and Hungarian}

\author{Dávid Márk Nemeskey \\
  Research Institute for \\ Computer Science and Automation \\
  Hungarian Academy of Sciences \\
  H-1111 Lágymányosi út 11, Budapest \\
  {\tt nemeskey.david@sztaki.hu} \\\And
  Eszter Simon \\
  Research Institute for Linguistics \\
  Hungarian Academy of Sciences \\
  H-1068 Benczúr utca 33, Budapest \\
  {\tt simon.eszter@nytud.mta.hu} \\}

\date{}

\begin{document}
\maketitle
\begin{abstract}

Supervised Named Entity Recognizers require large amount of annotated texts. However, manual annotation is a highly costly procedure, thus reducing the annotation cost is essential. We present a fully automatic method to build NE annotated corpora from Wikipedia. In contrast to recent work, we apply a new method, which maps the DBpedia classes into CoNLL NE types. Since our method is mainly language-independent, we used it for generating corpora for English and Hungarian, as well. The corpora are freely available. 

\end{abstract}

\section{Introduction}
\label{sec:intro}

Named Entity Recognition (NER), the task of identifying Named Entities (NEs) in unstructured texts and classifying them into pre-selected classes, is one of the most important subtasks in many NLP tasks, such as information retrieval, information extraction or machine translation. The NER task was introduced with the 6th Message Understanding Conference (MUC) in 1995 \cite{Grishman:96}. In MUC shared tasks the NER consists of three subtasks: entity names, temporal and number expressions. Although there is a general agreement in the NER community about the inclusion of temporal expressions and some numerical expressions, the most studied types are names of persons, locations and organizations. The fourth type, called "miscellaneous'', was introduced in the CoNLL NER tasks in 2002 \cite{Tjong:02} and 2003 \cite{Tjong:03}, and includes proper names falling outside the three classic types. Since then, MUC and CoNLL datasets and annotation schemes have been the major standards applied in the field of NER. 

The standard datasets are highly domain-specific (mostly newswire), and are restricted in size, as well. 
%Thus they are not large enough to train and test robust NE taggers on them. Moreover, 
Merging these datasets to get a bigger training corpus faces NLP researchers with the problem of combining different tagsets and annotation schemes. However, to build a robust supervised machine learning system for NER, large enough, accurately annotated corpora are essential. Annotating large amount of texts with linguistic information is a time-consuming, highly skilled and delicate work. Mainly due to this fact, one of the main challenges in NLP is reducing the annotation cost. 

One approach to reduce the manual intervention required in language resource production, and thus ultimately production costs is automatically generating resources. Another approach to achieve this aim is using collaborative annotation and/or collaboratively constructed resources, such as Wikipedia, Wiktionary or Linked Open Data. Researchers apply the Web-as-corpus approach, often dealing with such datasets as Google Web 1T 5-gram collection or DBpedia.  

%The first approach is the application of unsupervised and semi-supervised learning techniques. Researchers following this approach try to not only reduce, but avoid the annotation cost with minimizing the need for annotated data. These methods can be applied to any language or genre for which adequate raw text resources are available. 

In this paper we present a kind of combination of these approaches, namely automatically generating resources and using collaboratively constructed resources. We built freely available NE tagged corpora for English and Hungarian -- fully automatically generated from Wikipedia. %Our goal was to create gold standard corpora which can serve as train, devel and test datasets for automatic NE taggers. 

The paper is structured as follows. In Section~\ref{sec:related} we give an overview of related work. Section \ref{sec:create} contains a description of our method, while Section \ref{sec:hun} shows how it is applied to Hungarian. The corpus format is described in Section~\ref{sec:data}. In Section \ref{sec:eval} we present experiments and results on the newly generated datasets. Section \ref{sec:conclusion} concludes the paper with a summary. 

%Our method is introduced in Section~\ref{sec:create}, and showed how it is applied to Hungarian in Section~\ref{sec:hun}. The corpus format is described in Section~\ref{sec:data}. Section~\ref{sec:eval} contains evaluations and Section~\ref{sec:conclusion} our conclusion.

\section{Wikipedia and NER}
\label{sec:related}
% Wikify!

Wikipedia is a free, multilingual Internet encyclopedia, written collaboratively by volunteers\footnote{http://en.wikipedia.org/wiki/Wikipedia}. It is a goldmine of information; at the time of writing, Wikipedia contains about 21 million interlinked articles. Of these, 3,903,467 are English, and 212,120 are Hungarian. 

Wikipedia has been applied for several NLP tasks, e.g.~word sense disambiguation, ontology and thesaurus building, and question answering, as Medelyan et al.~\shortcite{Medelyan:09} point out in their fully detailed survey about mining meaning from Wikipedia. 

Wikipedia is recognized as one of the largest available collection of entities, and also as a resource that can improve the accuracy of NER. The most obvious utilization of Wikipedia for NER is extracting gazetteers containing person names, locations or organizations (e.g.~Toral and Mu\~noz \shortcite{Toral:06}). Creating dictionaries of entities is also a common step of NE disambiguation \cite{Bunescu:06,Cucerzan:07}. Not only supervised systems use lists generated from Wikipedia: Nadeau et al.~\shortcite{Nadeau:06} use similar lists in an unsupervised NER system. 

The knowledge in Wikipedia may be incorporated as features in NER learning, e.g.~Kazama and Torisawa \shortcite{Kazama:07} showed that automatic extraction of category labels from Wikipedia improves the accuracy of a supervised NE tagger. 

A third approach to improve NER with Wikipedia is the automatic creation of training data. Richman and Schone \shortcite{Richman:08} built corpora for less commonly taught languages annotated with NE tags. They used the inherent category structure of Wikipedia to determine the NE type of a proposed entity. Nothman et al.~\shortcite{Nothman:08} used a similar method to create a NE annotated text in English. They transformed the Wikipedia links into NE annotations by classifying the target articles into standard entity classes. Their approach to classification is based primarily on category head nouns and definitional opening sentences of articles. 

In contrast to the aforementioned methods, we applied a new approach to recognize and classify NEs in corpora generated from Wikipedia. We mapped the DBpedia ontology classes to standard NE tags and assigned them to Wikipedia entities (see more details in Section \ref{dbpedia}).

To our knowledge, such type of automatically built corpora are not freely available, with the one exception of the Semantically Annotated Snapshot of the English Wikipedia \cite{Zaragoza:07}. It provides a wide range of linguistic information: POS tags, dependency labels, WordNet super senses and NE annotation according to WSJ and CoNLL tagsets. However, they followed a different way: the NEs were tagged by high quality open source softwares. In contrast, our approach relies on the manual judgement of thousands of Wikipedia volunteers.

Thanks to the huge amount of Wikipedia articles, the possibility of building large enough 
corpora is given for less resourced languages, as well. Since our method is largely language-independent, it can be easily applied for languages which are structurally differing from English. Hungarian is a highly agglutinative language, with free word order and other particularities detailed later in Section \ref{sec:hun}. There are human annotated CoNLL-style datasets, but the one presented here is the first automatically NE annotated corpus for Hungarian.

\section{Creating the English Corpus} 
\label{sec:create}

Our goal is to create a large NE annotated corpus, automatically generated from Wikipedia articles. We follow a similar path to Nothman et al.~\shortcite{Nothman:08}, who broke down the process into four steps:

\begin{enumerate}
\item Classify all articles into entity classes.
\item Split Wikipedia articles into sentences.
\item Label NEs according to link targets.
\item Select sentences for inclusion in the corpus.
\end{enumerate}

%Subsectionöket valahogy ez alapján kéne.

In this section, we describe how these steps were implemented. While the approach in essence is language-independent, the exact details XXX have to be accomodated to the target language. 
%Ez utóbbi mondat mit jelent?

%This section explains the general idea and its execution for English; the next section describes how the idea is adapted to Hungarian.

\subsection{Parsing Wikipedia}

Wikipedia is a very rich source of information; aside from the article text, a huge amount of data is encompassed in infoboxes, templates and the category structure. Our task requires the use of only the links between the articles and the article text. In addition to in-article links, our method takes advantage of the redirect and interlanguage links, available as SQL dumps.

The English corpus is based on the Wikipedia snapshot as of January 15, 2011. The XML files were parsed by the mwlib parser\footnote{http://code.pediapress.com}. The raw text was tokenized by a modified version of the Punkt sentence and word tokenizers \cite{Kiss:06} in NLTK. 
%\cite{Loper:02}. 
Lemmatization was performed using the Wordnet Lemmatizer, also in NLTK. Part-of-speech tagging was performed by HunPos \cite{Halacsy:07}.
%The words were lemmatized with the Wordnet Lemmatizer, also in NLTK, and POS tagged with \texttt{HunPos} \cite{Halacsy:07}.

%Chunking was performed with HunChunk (CITE). Experiments have shown that HunPos obtains STAT f-score on the Brown corpus; HunChunk performs at around STAT, CITE?.

\subsection{Articles as Entities}
\label{dbpedia}

Many authors, such as Kazama and Torisawa \shortcite{Kazama:07} and Nothman et al.~\shortcite{Nothman:08} used semi-supervised methods based on Wikipedia categories and text to classify articles into NE types. To avoid the inevitable classification errors, we obtain entity type information from the DBpedia knowledge base \cite{Bizer:09}. This enables us to have high precision information about entity types at the expense of recall: of the 3,903,467 English Wikipedia pages, 1,470,293 are covered by DBpedia.

DBpedia is an open database that presents information about entities in Wikipedia in structured form. Each entity corresponds to a Wikipedia article. DBpedia provides various information about entities including type, properties, home pages, etc. Here we limit our interest to the type system.

The types in DBpedia are organized into a class hierarchy, available as an OWL\footnote{http://www.w3.org/TR/owl-ref/} ontology. The ontology contains the 320 most frequent entity categories, arranged into a taxonomy under the base class \texttt{owl:Thing}. Most of the classes belong to the 6 largest sub-hierarchies (here named after their topmost classes): \texttt{Person}, \texttt{Organisation}, \texttt{Event}, \texttt{Place}, \texttt{Species} and \texttt{Work}. The taxonomy is rather flat: the top level contains 44 classes and there are several nodes with a branching factor of 20.

The type of entities is extracted automatically from Wikipedia categories. However, the mapping between Wikipedia categories and classes in the DBpedia ontology is manually defined. This, together with the fact that the existence of the reference ontology prevents the proliferation of categories observable in Wikipedia \cite{Bizer:09}, ensures that type information in DBpedia can be considered gold quality.

From the available NER annotation standards we elected to use the CoNLL \cite{Tjong:03} NE types. It is not difficult to see the parallels between the DBpedia sub-hierarchies \texttt{Person}, \texttt{Organisation} and \texttt{Place} and the CoNLL NE types \texttt{PER}, \texttt{ORG} and \texttt{LOC}. The fourth category, \texttt{MISC} is more elusive; according to the CoNLL NER annotation guide\footnote{http://www.cnts.ua.ac.be/conll2003/ner/annotation.txt}, the sub-hierarchies \texttt{Event} and \texttt{Work} belong to this category, as well as various other classes outside the main hierarchies. 

While the correspondence described above holds for most classes in the sub-hierarchies, there are some exceptions. For instance, the class \texttt{SportsLeague} is part of the \texttt{Organisation} sub-hierarchy, but according to the CoNLL annotation scheme, they should be tagged as \texttt{MISC}. 

To avoid such kind of misclassification, 
%as well as the miscellaneous classes outside the main hierarchies, we have introduced a mapping mechanism. 
we created a file of DBpedia class--NE category mappings. Whenever an entity is evaluated, we look up its class and the ancestors of its class, and assign to it the category of the class that matches it most closely. If no match is found, the entity is tagged with \texttt{O}. 
%Since we take advantage of the inheritance hierarchy, the mapping list remains short: it contains only the root classes of the main hierarchies, exceptions like those mentioned above, and the various classes that belong to the MISC category according to the CoNLL annotation guideline.

As of version 3.7, the DBpedia ontology allows multiple superclasses, making a directed acyclic graph\footnote{http://blog.dbpedia.org/2011/09/11/dbpedia-37-released-including-15-localized-editions}. Since selecting the right superclass, and hence, CoNLL tag for classes with more than one parent cannot be reliably done automatically, the class-to-category mapping has to be determined manually. The only such class in version 3.7 (\texttt{Library}) can be traced back to both \texttt{Place} and \texttt{Organisation}; its CoNLL tag is \texttt{LOC}.

%The full mapping can be found in Appendix~\ref{sec:appa}.

Using the mapping described above, we compile a list that contains all entities in DBpedia tagged with the appropriate CoNLL category.

We would like to note that even though we use CoNLL categories, our method can be modified to work with any tagset compatible with the DBpedia ontology. Indeed, the DBpedia classes define a NE tagset themselves. We leave the exploration of these possibilities for future work.

\subsection{Named Entity Labeling}

In order to automatically prepare sentences where NEs are accurately tagged, two tasks need to be performed: identifying entities in the sentence and tagging them with the correct tag. Sentences, for which accurate tagging could not be accomplished, must be removed from the corpus.

Our approach to entity identification relies on Wikipedia cross-references found in the article text. We assume that individual Wikipedia articles describe NEs. A link to an article can then be perceived as a mapping that identifies its anchor text with a particular NE.

Discovered entities are tagged with the CoNLL label assigned to them in the entity list extracted from DBpedia. If the link target is not in the entity list, or the link points to a disambiguation page, we can not determine the type of the entity, and tag it as UNK. Links to redirect pages are resolved to point instead to the redirect target, after which they are handled as regular cross-references. Finally, sentences with UNK links in them are removed from the corpus.

The following sub-sections describe how the method explained above can be improved to increase precision, sentence coverage and to account for peculiarities in the CoNLL guidelines.

\subsubsection{Non-entity Links}

Strictly speaking, our original assumption of equating Wikipedia articles with NEs is not valid: many pages describe common nouns (Book, Aircraft), calendar-related concepts (March 15, 2007), or other concepts that fall outside the scope of NER. To increase sentence coverage, we modified the algorithm to prevent it from misclassifying links to these pages as unknown entities and discarding the sentence.

\begin{description}
\item[Common nouns] Links to common nouns are filtered by POS tags; if a link contains no \texttt{NNP}s, it is ignored.
\item[Dates] Time expressions require special attention, because dates and months are often linked to the respective Wikipedia pages. We circumvented this problem by compiling a list of calendar-related pages and adding them to the main entity list tagged with the CoNLL category \texttt{O}. 
\item[Lowercase links] Entities are sometimes referred to by common nouns, such as \textit{republic} to \textit{Roman Republic}, \texttt{LOC}, which results in a lowercase link. These references are not considered NEs and are ignored.
\end{description}

\subsubsection{Unmarked Entities}

Not all entities are marked with links. Authors may neglect to add references to their text. Furthermore, in a Wikipedia article, typically only the first occurrence of a particular entity is linked to the corresponding page. Subsequent mentions are unmarked and often incomplete -- e.g.~family names are used instead of full names. To account for such mentions, we try to associate capitalized words in the article text with entities we have tagged thus far.

For each article, we build a set that contains all entities discovered so far, as well as their aliases:
\begin{itemize}
\item If the current page is a NE:
  \begin{itemize}
  \item The page title, with Wikipedia antics, such as a category name in parentheses removed
  \item Title of redirect pages that target the current page
  \item Title of the current page in other languages, as Wikipedia usually mentions the original name of foreign entities
  \end{itemize}
\item Entities marked with links in the text thus far
\item Certain subsets of the above:
  \begin{itemize}
  \item First and last names of \texttt{PER} entities
  \item Numbers, such as \textit{737} in \textit{Boeing 737}
  \end{itemize}
\end{itemize}

When an unmarked entity is encountered, the set is consulted to find an item that contains it in its entirety; i.e.~all of its words, in the same order. If either exactly one such item is found, or the matching entities in the list all have the same NE tag, the unmarked entity is tagged accordingly; otherwise it is tagged as \texttt{UNK}.

Our solution is similar to Nothman's \shortcite{Nothman:08}. Instead of a trie, however, we use a set, albeit with more alias types. We expect more precise tagging from our slightly more rigorous solution.

\subsubsection{Special Cases}

%There are some special cases to the method, which are detailed below.

\begin{description}
\item[Derived words] According to the CoNLL guidelines, words derived from NEs are tagged as \texttt{MISC}. We complied with this rule by tagging each entity whose head is not a noun, as well as when the link's anchor text is not contained in the entity's name, as \texttt{MISC}. The most prominent example for such entities are nationalities, which can be linked to their home country, a \texttt{LOC}; e.g.~\textit{Turkish} to \textit{Turkey}. Our solution assigns the correct tag to these entities.
\item[First word in a sentence] As first words are always capitalized, labeling them is difficult if they are unlinked and not contained in the entity alias set. We base the decision on the POS tag of the first word: if it is \texttt{NNP}, we tag it as \texttt{UNK}; otherwise, \texttt{O}.
\item[Reference cleansing] Page titles and anchor texts may contain more than just the entity name. Personal titles are part of the entity name in Wikipedia, but not in CoNLL, and punctuation marks around the entity may become part of the link by mistake. We tag all punctuation marks after the entity name as \texttt{O}.

To tackle personal titles, we extracted a list from the Wikipedia page \textit{List of titles}, which contains titles in many languages. We manually removed all titles that also function as given names, such as \textit{Regina}. If a link to a \texttt{PER} or \texttt{UNK} entity, or an unlinked entity starts with, or consists solely of a title in the list, we tag the words that make up the title as \texttt{O}.
\item[Incidental capitalization] Various non-\texttt{NNP} words in English are capitalized: names of months, the \textit{I} pronoun and non-entity acronyms, such as \textit{RSVP}. While the latter two group is unlikely to appear in Wikipedia text, we assembled a list of these words and tag them as \texttt{O} unless they are part of the alias set.
\end{description}

\subsubsection{Sentence Filtering}

As mentioned above, sentences with words tagged as \texttt{UNK} are discarded. Furthermore, there are many incomplete sentences in the Wikipedia text: image captions, enumerations items, contents of table cells, etc. On the one hand, these sentence fragments may prove to be of too low quality to be of any use in the traditional NER task. On the other hand, they could prove to be invaluable when training a NER tagger for User Generated Content, which is known to be noisy and fragmented. As a compromise solution, we included these fragments in the corpus, but labelled them as "low quality", so that users of the corpus can decide whether they want to use them or not. A sentence is labelled as such if it either lacks a punctuation mark at the end, or it contains no finite verbs.

\section{Creating the Hungarian Corpus}
\label{sec:hun}

The procedure described in the previous section was used to generate the Hungarian corpus, as well. However, linguistic diversity pose several problems. In this section we describe the differences between the two languages related to labeling NEs, and the changes they prompted in the method.

\begin{table*}[ht]
\begin{center}
\begin{tabular}{lllll}
\hline \bf train & \bf test & \bf precision & \bf recall & \bf F-measure \\ \hline
Szeged NER & Szeged NER & 94.50 & 94.35 & 94.43 \\
huwiki & huwiki & 90.64 & 88.91 &  89.76 \\
huwiki & Szeged NER & 63.08 & 70.46 & 66.57 \\
Szeged NER with wikilists & Szeged NER & 95.48 & 95.48 & 95.48 \\
Szeged NER with wikitags & Szeged NER & 95.38 & 94.92 & 95.15 \\
\hline
\end{tabular}
\end{center}
\caption{\label{huresults} Hungarian results.}
\end{table*}

\subsection{Parsing the Hungarian Wikipedia}

Although Hungarian is reckoned to be a less resourced language, and it is not supperted in NLTK, several high quality language processing tools have been developed for Hungarian in recent years. For tokenization and sentence segmentation we used an in-house statistical tool tailored for Hungarian. It has been trained on the largest manually annotated Hungarian corpus \cite{Csendes:04}, and it handles the peculiarities of Hungarian ortography, such as the periods placed after numbers in date expressions. Lemmatization was performed by \texttt{hunmorph} \cite{Tron:05} and \texttt{hundisambig}, the latter being an in-house disambiguator to select the right analysis based on the word context. 

%Although HunPos, as its name implies, is fully capable of POS tagging Hungarian text, we opted for Hunmorph (CITE), a morphological analyzer that, apart from PAS tagging, also lemmatizes the words. Hundisambig (CITE) was used to select the right analysis based on the word context\footnote{Incidentally, Hundisambig relies on HunPos for its HMM model.}.

Hungarian is called a synthetic language: for the most part, it expresses grammatical elements in a single word form using affixes, as opposed to isolating languages (as English), which tend to employ separate words, e.g.~prepositions, pronouns, auxiliaries, for expressing grammatical phenomena. The Hungarian morphological analyzer outputs KR-codes \cite{Kornai:04}, which, in addition to the POS category, also includes inflectional information, which makes it much better suited to agglutinative languages, than Penn Treebank POS tags. One shortcoming of the KR-code is that it does not differenciate between common and proper nouns. Since in Hungarian only proper nouns are capitalized, we can usually decide whether a noun is proper or not based on the initial letter. However, this rule cannot be used if the noun is at the beginning of a sentence, as words at the beginning of sentences are always capitalized. Hence, all sentences that begin with nouns have been removed from the corpus.

\subsection{Named Entity Labeling in Hungarian}

For well-resourced languages, DBpedia has internationalized chapters; data dumps based on Wikipedia. Hungarian does not have a separate chapter. Instead, the Hungarian entity list comprises of the pages in the English list that have their equivalents in the Hungarian Wikipedia. Two consequences follow.

First, in order to identify which pages denote entities in the Hungarian Wikipedia, an additional step is required, in which the Hungarian equivalents of the English pages are added to the entity list. The English titles are retained, as due to the medium size of the Hungarian Wikipedia, in-article links sometimes point to English articles.

Second, entities without a page in the English Wikipedia are absent from the entity list. This gives rise to two potential problems. One is that compared to English, the list is relatively shorter: the entity/page ratio is 12.12\%, as opposed to the 37.66\% of the English Wikipedia. The other, since mostly Hungarian people, places and organizations are missing, a NER tagger that takes the surface forms of words into account might be mislead as to the language model of entity names. To overcome these problems, the list has to be extended with Hungarian entity pages that do not have a corresponding English page. We leave this for future work.

To annotate our corpus with NE tags, we chose to follow the annotation guidelines of the largest human-annotated NER corpus for Hungarian, Szeged NER corpus \cite{Szarvas:06}. It is similar to CoNLL standards: contains newswire texts, comprises ca.~200,000 tokens, and is annotated with NE class labels in line with the CoNLL annotation scheme. However, the convention of what constitutes a NE is slightly different for Hungarian. 

\subsubsection{Special cases}

The Szeged NER guideline relies heavily on the rules of capitalization to decide which words should be marked as NEs. As mentioned earlier, in Hungarian, apart from the first word in the sentence, only proper nouns are capitalized. The following concepts are not proper nouns in Hungarian, and thus are not considered as NEs according to the Szeged NER guideline: names of languages, nationalities, religions, political ideologies; adjectives derived from a NE; names of months, days, holidays; names of special events and wars.

%\begin{itemize}
%\item names of languages, nationalities, religions, political ideologies;
%\item adjectives derived from a NE;
%\item names of months, days, holidays;
%\item names of special events and wars.
%\end{itemize}

There is another special case in Hungarian: opposite to English, the number of compund words is quite large and NEs can also be subjects of compounding. In this case the common noun following the NE is joint with a hyphen, so they constitute one token. However, the joint common noun can modify the original sense of NE, depending on the semantics of the common noun. For example: in the compound \textit{SAP-szoftvercég} ['SAP software company'] the common noun does not change the labeling, but in the case of the compound \textit{WorldCom-botrány} ['WorldCom scandal'] the NE tag changes from ORG to O. The solution of this problem is not obvious, and needs more investigation.

% TODO: <NE>-<random kisbetűs szó> => nem NE (magyar). Lásd: "a tű nélküli mezoterápia alapja a <Nobel, PER>-díjas"

%Inflections of acronyms and foreign names ending with non-pronounced vowel have similar surface forms to the aforementioned compounds, e.g.~\textit{MTI-t} ['MTI'+ACC], \textit{Shakespeare-rel} ['with Shakespeare']. It is important to distinguish these types of hyphenated NEs, because inflections do not change NE labeling in contrast to some type of compounds. Due to the surface similarity of these phenomena the solution of this problem is not obvious, and needs more investigation. 

\section{Data Description}
\label{sec:data}

The corpora are available under the Creative Commons Attribution-Sharealike 3.0 Unported License (CC-BY-SA), the same license under which the text of Wikipedia is released. The data files can be freely downloaded from ...\footnote{To be filled in in the camera-ready version.}.

The files are in multitag format. Content lines are tab separated; there is one column for the tokens plus one column per tagset. Sentence boundaries are marked by empty lines. The linguistic features include the lemmatized form of the word and its part-of-speech tag. Two NE tags are included with each word: the most specific DBpedia category it belongs to and the CoNLL NE tag. While the NE tags can be considered as a "silver standard", the linguistic features are provided on a "best-effort" basis.

\begin{table*}[ht]
\begin{center}
\begin{tabular}{lllllll}
\hline  & \bf enwiki & \bf enwiki filtered & \bf CoNLL & \bf huwiki & \bf huwiki filtered  & \bf Szeged NER \\ \hline
token & 60,520,819 & 21,718,854 & 302,811 &  19,108,027 & 3,512,249  & 225,963\\
NE & 3,169,863 & 3,169,863 & 50,758 & 456,281 & 456,281  & 25,896\\
NE density & 5.23\% & 14.59\% & 16.76\% & 2.38\% & 12.99\%  & 11.46\%\\
\hline
\end{tabular}
\end{center}
\caption{\label{size} Corpus size and NE density.}
\end{table*}

\begin{table*}[ht]
\begin{center}
\begin{tabular}{lllll}
\hline \bf train & \bf test & \bf precision & \bf recall & \bf F-measure \\ \hline
CoNLL & CoNLL & 85.13 & 85.13 & 85.13 \\
enwiki & enwiki & 72.46 & 73.33 &  72.89 \\
enwiki & CoNLL & 56.55 & 49.77 & 52.94 \\
CoNLL with wikilists & CoNLL & 86.33 & 86.35 & 86.34 \\
CoNLL with wikitags & CoNLL & 85.88 & 85.94 & 85.91 \\
\hline
\end{tabular}
\end{center}
\caption{\label{enresults} English results.}
\end{table*}

\section{Evaluation}
\label{sec:eval}

Having the obvious advantages, an automatically generated corpus can not serve as a gold standard dataset. Then what can we do with silver standard corpora? They can be very useful for improving NER in more ways: (a) for less resourced languages, they can serve as training corpora in absence of gold standard datasets; (b) they can serve as (supplementary or independent) training sets for domains differing from newswire; (c) they can be sources of huge entity lists, and (d) feature extraction. 

To evaluate our corpora we used a maximum entropy NE tagger \cite{Varga:07}, which was originally developed for labeling NEs in Hungarian texts, but it can be tuned for different languages, as well. Corpus-specific features (e.g.~NP chunks, Wikipedia links) were removed to get better comparability, so the feature set consists of gazetteer features; sentence start and end position; Boolean-valued orthographic properties of the word form; string-valued surface properties of the word form; and morphological information.

%\begin{itemize}
%\item gazetteer features;
%\item sentence start and end position;
%\item Boolean-valued orthographic properties of the word form;
%\item string-valued surface properties of the word form;
%\item morphological information.
%\end{itemize}

We used the CoNLL standard method for evaluation. According to this, an automatic labeling is correct if it gives the same start and end position, and the same NE class as the gold standard. Based on this, precision and recall can be calculated, and the F-measure, as usual, the harmonic mean of these two values.  

\subsection{Wikipedia data}

Our automatic annotation process retains all of the Wikipedia sentences, which remained after our two-fold filtering method. Thus sentences without NEs are also included in the corpus. The rationale behind this is that we wanted to reserve the original distribution of names in Wikipedia as much as possible. However, after further investigation of the NE density in our corpora and gold standard corpora (see Table \ref{size}), we decided not to include the sentences without NEs in evaluation datasets. 

\subsection{Experiments and results}

The English train corpus was evaluated against itself and a manually annotated English corpus. Since the filtered English Wikipedia corpus, containing only the sentences with NEs, is still very huge, and we are limited by time and memory, our experiments were performed with a portion of 3.5 million tokens (the size of our filtered Hungarian corpus). It was divided into train and test sets (90\%-10\%). 

For English cross-corpus evaluation the CoNLL-2003 corpus was chosen. It is well-studied that cross-checking between different corpora decrease F-measure. Domain differences certainly affect NER performance, and the different annotation schemes pose several compatibility problems. Nothman et al.~\shortcite{Nothman:08} showed that each set of gold standard training data performs better on corresponding test sets than on test sets from other sources. The situation here is similar: the NE tagger trained on Wikipedia does not achieve as high performance tested against CoNLL test set as one trained on the own train set (see Table \ref{enresults}). 

However, Wikipedia-derived corpora can also be used for improving NER accuracy in other ways. First, we collected gazetteer lists from the corpus for each NE category, which improved the overall F-measure given to the NE tagger training and testing on CoNLL dataset. A second trial was labeling the CoNLL datasets by the model trained on Wikipedia corpus, and giving these labels as extra features to the next CoNLL train. Both methods result in improved F-measure on CoNLL test set.  

Since in Hungarian NE tagging we followed the Szeged NER corpus annotation guidelines, we performed the experiments on this dataset. Hungarian results are similar to the English ones (see Table \ref{huresults}), the only difference is that F-measures for Hungarian are significantly higher. It can be due to the fact that the MISC category for Hungarian contains less types of names, thus the inconsistency of this class is smaller (see section \ref{sec:hun}). In contrast to the CoNLL corpus, the Szeged NER corpus was accurately annotated with an inter-annotator agreement over 99\%.  

Due to the quite good F-measure of training on our Hungarian train corpus and testing on the corresponding test set, our Hungarian corpus can serve as a training corpus to build NE taggers for domains falling out of newswire. 

\section{Conclusion}
\label{sec:conclusion}

We have presented freely available NE tagged corpora for English and Hungarian, fully automatically generated from Wikipedia. In contrast to the methods used so far for automatic annotation of NEs in Wikipedia texts, we applied a new approach, namely mapping DBpedia ontology classes to standard CoNLL NE tags, and assigning them to Wikipedia entities. Following Nothman \shortcite{Nothman:08}, the process can be divided into four main steps:  classifying all articles into entity classes; splitting Wikipedia articles into sentences; labeling NEs according to link targets; and selecting sentences for inclusion in the corpus. 

Thanks for the huge amount of Wikipedia articles, the possibility of building large enough corpora is given for less resourced languages, like Hungarian, as well. Due to the particularities of Hungarian, some steps are slightly different, and special linguistics phenomena pose several problems related to the NER task to solve. 

Automatically generated corpora can be useful for improving NER in more ways. We showed that gazetteer lists extracted from our corpora, and training with extra features given by the model trained on our corpora, improves F-measure. Moreover, our Hungarian corpus can serve as a training corpus for more general domains. 
%than the classic newswire.  

%az eredeti DBpedia osztályokat is megtarthatnánk, hiszen nem mindig célravezető a CoNLL osztályok használata, mivel azok elég tág kategóriák, amikbe olyan dolgok is tartoznak, amik nagyon eltérőek is lehetnek

%később majd kirakjuk a scripteket is, és akár újra és újra frissíthető lesz

\section*{Acknowledgments}

% TODO: the corpora will be published in the META-SHARE repository.
This research was supported by the CESAR project under the ICT Policy Support Programme (grant 271022). The authors are grateful to Attila Zséder for his work on Wikipedia parsing.

\begin{thebibliography}{}

%\bibitem[\protect\citename{Balasuriya \bgroup et al.\egroup}2009]{Balasuriya:09}
%Dominic Balasuriya, Nicky Ringland, Joel Nothman, Tara Murphy, James R. Curran
%\newblock 2009.
%\newblock Named Entity Recognition in Wikipedia.
%\newblock In: {\em Proceedings of the 2009 Workshop on the People's Web Meets NLP, ACL-IJCNLP}, pages 10--18.

\bibitem[\protect\citename{Bizer \bgroup et al.\egroup}2009]{Bizer:09} 
Christian Bizer, Jens Lehmann, Georgi Kobilarov, Sören Auer, Christian Becker, Richard Cyganiak, Sebastian Hellmann.
\newblock 2009.
\newblock DBpedia -- A Crystallization Point for the Web of Data. 
\newblock In: {\em Journal of Web Semantics: Science, Services and Agents on the World Wide Web}, Issue 7, pages 154--165.

\bibitem[\protect\citename{Bunescu and Pasca}2006]{Bunescu:06}
B. Bunescu and M. Pasca.
\newblock 2006.
\newblock Using encyclopedic knowledge for named entity disambiguation.
\newblock In: {\em Proceedings of the 11th Conference of the European Chapter of the Association for Computational Linguistics}, pages 9--16.

\bibitem[\protect\citename{Csendes \bgroup et al.\egroup}2004]{Csendes:04} 
Dóra Csendes, János Csirik, Tibor Gyimóthy.
\newblock 2004.
\newblock The Szeged Corpus: A POS tagged and Syntactically Annotated Hungarian Natural Language Corpus. 
\newblock In: {\em Proceedings of TSD 2004}, vol. 3206, pages 41--49.

\bibitem[\protect\citename{Cucerzan}2007]{Cucerzan:07}
S. Cucerzan.
\newblock 2007.
\newblock Large-scale named entity disambiguation based on Wikipedia data.
\newblock In: {\em Proceedings of the 2007 Joint Conference on Empirical Methods in Natural Language Processing and Computational Natural Language Learning}.
\newblock Prague, Czech Republic, June 2007. pages 708--716.

\bibitem[\protect\citename{Grishman and Sundheim}1996]{Grishman:96}
Ralph Grishman and B. Sundheim.
\newblock 1996.
\newblock Message Understanding Conference -- 6. 
\newblock In: {\em Proc. International Conference on Computational Linguistics}.

\bibitem[\protect\citename{Halácsy \bgroup et al.\egroup}2007]{Halacsy:07}
P. Halácsy, A. Kornai and Cs. Oravecz.
\newblock 2007.
\newblock Hunpos -- an open source trigram tagger. 
\newblock In: {\em Proceedings of the 45th Annual Meeting of the Association for Computational Linguistics}, pages 209--212.

\bibitem[\protect\citename{Kazama and Torisawa}2007]{Kazama:07}
Jun'ichi Kazama and Kentaro Torisawa.
\newblock 2007.
\newblock Exploiting Wikipedia as External Knowledge for Named Entity Recognition.
\newblock In: {\em Proceedings of the 2007 Joint Conference on Empirical Methods in Natural Language Processing and Computational Natural Language Learning}, pages 698--707.

\bibitem[\protect\citename{Kiss and Strunk}2002]{Kiss:06}
Tibor Kiss, Jan Strunk.
\newblock 2006.
\newblock Unsupervised Multilingual Sentence Boundary Detection. 
\newblock In: {\em Computational Linguistics}, 32 (4): pages 485--525.

\bibitem[\protect\citename{Kornai \bgroup et al.\egroup}2004]{Kornai:04}
András Kornai, Péter Rebrus, Péter Vajda, Péter Halácsy, András Rung, Viktor Trón.
\newblock 2004.
\newblock Általános célú morfológiai elemző kimeneti formalizmusa (The output formalism of a general-purpose morphological analyzer).
\newblock In: {\em Proceedings of the 2nd Hungarian Computational Linguistics Conference}.

\bibitem[\protect\citename{Loper and Bird}2002]{Loper:02}
Edward Loper, Steven Bird.
\newblock 2002.
\newblock NLTK: the Natural Language Toolkit. 
\newblock In: {\em Proceedings of the ACL-02 Workshop on Effective tools and methodologies for teaching natural language processing and computational linguistics}, Volume 1, pages 63--70.

\bibitem[\protect\citename{Medelyan  \bgroup et al.\egroup}2009]{Medelyan:09}
Olena Medelyan, David Milne, Catherine Legg, and Ian H. Witten.
\newblock 2009.
\newblock Mining meaning from Wikipedia.
\newblock {\em International Journal of Human-Computer Studies}, 67: 716--754.

\bibitem[\protect\citename{Nadeau  \bgroup et al.\egroup}2006]{Nadeau:06} 
David Nadeau, Peter D. Turney and Stan Matwin.
\newblock 2006.
\newblock Unsupervised named entity recognition: Generating gazetteers and resolving ambiguity. 
\newblock In: {\em Proceedings of the 19th Canadian Conference on Artificial Intelligence}, volume 4013 of {\em LNCS}, pages 266--277.

\bibitem[\protect\citename{Nothman \bgroup et al.\egroup}2008]{Nothman:08}
Joel Nothman, James R. Curran, and Tara Murphy.
\newblock 2008.
\newblock Transforming Wikipedia into Named Entity Training Data.
\newblock In: {\em Proceedings of the Australasian Language Technology Workshop}, Vol 6., pages 124--132.

\bibitem[\protect\citename{Nothman \bgroup et al.\egroup}2009]{Nothman:09}
Joel Nothman and Tara Murphy and James R. Curran.
\newblock 2009.
\newblock Analysing Wikipedia and Gold-Standard Corpora for NER Training.
\newblock In: {\em Proceedings of the 12th Conference of the European Chapter of the ACL}, pages 612--620.

\bibitem[\protect\citename{Richman and Schone}2008]{Richman:08}
Alexander E. Richman and Patrick Schone.
\newblock 2008.
\newblock Mining Wiki Resources for Multilingual Named Entity Recognition.
\newblock In: {\em Proceedings of ACL-08: HLT}, pages 1--9.

\bibitem[\protect\citename{Szarvas \bgroup et al.\egroup}2006]{Szarvas:06}
György Szarvas, Richárd Farkas, András Kocsor.
\newblock 2006.
\newblock A highly accurate Named Entity corpus for Hungarian. 
\newblock In: {\em Proceedings of International Conference on Language Resources and Evaluation}.

\bibitem[\protect\citename{Tjong Kim Sang}2002]{Tjong:02}
Erik F. Tjong Kim Sang.
\newblock 2002.
\newblock Introduction to the CoNLL-2002 shared task: Language-independent named entity recognition.
\newblock In: {\em Proceedings of the 6th Conference on Natural Language Learning}, pages 1--4, Taipei, Taiwan.

\bibitem[\protect\citename{Tjong Kim Sang and De Meulder}2003]{Tjong:03}
Erik F. Tjong Kim Sang and Fien De Meulder.
\newblock 2003.
\newblock Introduction to the CoNLL-2003 shared task: Language-independent named entity recognition. 
\newblock In: {\em Proceedings of the 7th Conference on Natural Language Learning}, pages 142--147, Edmonton, Canada.

\bibitem[\protect\citename{Toral and Mu\~noz}2006]{Toral:06}
A. Toral and R. Mu\~noz.
\newblock 2006.
\newblock A proposal to automatically build and maintain gazetteers for Named Entity Recognition by using Wikipedia.
\newblock In: {\em EACL 2006}.

\bibitem[\protect\citename{Trón \bgroup et al.\egroup}2005]{Tron:05}
Viktor Trón, György Gyepesi, Péter Halácsy, András Kornai, László Németh, Dániel Varga.
\newblock 2005.
\newblock Hunmorph: open source word analysis.
\newblock In: {\em Proceedings of the ACL 2005 Workshop on Software}. 

\bibitem[\protect\citename{Varga and Simon}2007]{Varga:07}
Dániel Varga and Eszter Simon.
\newblock 2007.
\newblock Hungarian named entity recognition with a maximum entropy approach.
\newblock {\em Acta Cybernetica}, 18: 293--301.

\bibitem[\protect\citename{Zaragoza \bgroup et al.\egroup}2007]{Zaragoza:07}
H. Zaragoza and J. Atserias and M. Ciaramita and G. Attardi.
\newblock 2007.
\newblock Semantically Annotated Snapshot of the English Wikipedia v.1 (SW1).
\newblock \url{http://www.yr-bcn.es/semanticWikipedia}


\end{thebibliography}

%\appendix
%\section{DBpedia to CoNLL Mapping}
%\label{sec:appa}
%
%\begin{table*}[ht]
%\begin{center}
%\begin{tabular}{llll}
%\hline DBpedia & \bf CoNLL & \bf DBpedia & \bf CoNLL \\ \hline
%Person & PER & Library & LOC \\
%Place & LOC & MeanOfTransportation & MISC \\
%Organization & ORG & PeriodicalLiterature & ORG \\
%Organisation & ORG & ProgrammingLanguage & MISC \\
%Award & MISC & Project & MISC \\
%EthnicGroup & MISC & SportsLeague & MISC \\
%Event & MISC & SportsTeamSeason & O \\
%Holiday & MISC & Weapon & MISC \\
%Ideology & MISC & Work & MISC \\
%Language & MISC & & \\
%\hline
%\end{tabular}
%\end{center}
%\end{table*}

\end{document}

% Jobban ki kell hangsúlyozni az introductionben, és a related worksben, hogy szabadon elérhető training korpuszt építettünk
% És azt, hogy Hungarian middle, English nagy.
