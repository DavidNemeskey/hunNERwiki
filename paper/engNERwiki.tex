%
% File acl2012.tex
%
% Contact: Maggie Li (cswjli@comp.polyu.edu.hk), Michael White (mwhite@ling.osu.edu)
%%
%% Based on the style files for ACL2008 by Joakim Nivre and Noah Smith
%% and that of ACL2010 by Jing-Shin Chang and Philipp Koehn

\documentclass[11pt]{article}
\usepackage[english]{babel}
\usepackage{acl2012}
\usepackage{times}
\usepackage{latexsym}
\usepackage{amsmath}
\usepackage{multirow}
\usepackage{url}
\usepackage[utf8]{inputenc}
\DeclareMathOperator*{\argmax}{arg\,max}
\setlength\titlebox{6.5cm}    % Expanding the titlebox
\hyphenation{DBpedia}

\title{Automatically generated NE tagged corpora for English and Hungarian}

\author{Eszter Simon \\
  Research Institute for Linguistics \\
  Hungarian Academy of Sciences \\
  H-1068 Benczúr utca 33, Budapest \\
  {\tt simon.eszter@nytud.mta.hu} \\\And
  Dávid Márk Nemeskey \\
  Research Institute for \\ Computer Science and Automation \\
  Hungarian Academy of Sciences \\
  H-1111 Kende utca 13-17, Budapest \\
  {\tt nemeskey.david@sztaki.mta.hu} \\}

\date{}

\begin{document}
\maketitle
\begin{abstract}

Supervised Named Entity Recognizers require large amounts of annotated
text. Since manual annotation is a highly costly procedure, reducing the
annotation cost is essential. We present a fully automatic method to build NE
annotated corpora from Wikipedia. In contrast to recent work, we apply a new
method, which maps the DBpedia classes into CoNLL NE types. Since our method
is mainly language-independent, we used it to generate corpora for English
and Hungarian. The corpora are freely available.

\end{abstract}

\section{Introduction}
\label{sec:intro}

Named Entity Recognition (NER), the task of identifying Named Entities (NEs) in unstructured texts and classifying them into pre-selected classes, is one of the most important subtasks in many NLP tasks, such as information retrieval, information extraction or machine translation. The NER task was introduced with the 6th Message Understanding Conference (MUC) in 1995 \cite{Grishman:96}. In MUC shared tasks the NER consists of three subtasks: entity names, temporal and number expressions. Although there is a general agreement in the NER community about the inclusion of temporal expressions and some numerical expressions, the most studied types are names of persons, locations and organizations. The fourth type, called ``miscellaneous'', was introduced in the CoNLL NER tasks in 2002 \cite{Tjong:02} and 2003 \cite{Tjong:03}, and includes proper names falling outside the three classic types. Since then, MUC and CoNLL datasets and annotation schemes have been the major standards applied in the field of NER. 

The standard datasets are highly domain-specific (mostly newswire) and are
restricted in size. Researchers attempting to merge these datasets to get a bigger training corpus are faced with the
problem of combining different tagsets and annotation schemes. Manually
annotating large amounts of text with linguistic information is a
time-consuming, highly skilled and delicate job, but large, accurately
annotated corpora are essential for building robust supervised machine
learning NER systems. Therefore, reducing the annotation cost is a key
challenge.

One approach is to generate the resources automatically, another one is to use
collaborative annotation and/or collaboratively constructed resources, such as
Wikipedia, Wiktionary, Linked Open Data, or DBpedia. In this paper we combine
these approaches by automatically generating freely available NE tagged
corpora from Wikipedia.

The paper is structured as follows. In
Section~\ref{sec:related} we give an overview of related work. Section
\ref{sec:create} contains a description of our method, and Section
\ref{sec:hun} shows how it is applied to Hungarian. The corpus format is
described in Section~\ref{sec:data}. In Section \ref{sec:eval} we present
experiments and results on the newly generated datasets. Section
\ref{sec:conclusion} concludes the paper with a summary.

\section{Wikipedia and NER}
\label{sec:related}

Wikipedia (WP, see {\tt http://wikipedia.org}), a free
multilingual Internet encyclopedia, written collaboratively by volunteers, is
a goldmine of information: at the time of writing, WP contains about 21
million interlinked articles. Of these, 3,903,467 are English, and 212,120 are
Hungarian. WP has been applied to several NLP tasks such as word sense
disambiguation, ontology and thesaurus building, and question answering (see
Medelyan et al.~\shortcite{Medelyan:09} for a survey). It is recognized as one
of the largest available collections of entities, and also as a resource that
can improve the accuracy of NER. The most obvious utilization of WP for NER is
extracting gazetteers containing person names, locations or organizations
(e.g.~Toral and Mu\~noz \shortcite{Toral:06}). Creating dictionaries of
entities is also a common step of NE disambiguation
\cite{Bunescu:06,Cucerzan:07}. Both supervised and unsupervised NER systems
use such lists, see e.g.~Nadeau et al.~\shortcite{Nadeau:06} The knowledge
embodied in WP may also be incorporated in NER learning as features,
e.g.~Kazama and Torisawa \shortcite{Kazama:07} showed that automatic
extraction of category labels from WP improves the accuracy of a supervised NE
tagger.

Another approach to improve NER with WP is the automatic creation of training
data. Richman and Schone \shortcite{Richman:08} built corpora for less
commonly taught languages annotated with NE tags. They used the inherent
category structure of WP to determine the NE type of a proposed
entity. Nothman et al.~\shortcite{Nothman:08} used a similar method to create
a NE annotated text in English. They transformed the WP links into NE
annotations by classifying the target articles into standard entity
classes. Their approach to classification is based primarily on category head
nouns and the opening sentences of articles where definitions are often given.

Our approach to recognize and classify NEs in corpora generated from WP was to
map the DBpedia ontology classes to standard NE tags and assign these to WP
entities (see more details in Section \ref{dbpedia}). Except for the
Semantically Annotated Snapshot of the English WP (SASWP) \cite{Zaragoza:07},
no such automatically built corpora are freely available. SASWP provides a
wide range of linguistic information: POS tags, dependency labels, WordNet
super senses and NE annotation according to WSJ and CoNLL tagsets. Even though
the SASWP NEs were tagged by the best available open source taggers, the tags 
provided here, being based on the manual judgement of thousands of WP
volunteers, are more reliable. Given the huge number of WP articles we can
build sufficiently large corpora for less resourced languages as well, as our
method is largely language-independent. We demonstrate this on Hungarian, a
highly agglutinative language, with free word order and other typological 
characteristics detailed later in Section \ref{sec:hun}. There are smaller, manually annotated CoNLL-style datasets, but the one presented here is the first automatically NE annotated corpus for Hungarian.

\section{Creating the English Corpus} 
\label{sec:create}

Our goal is to create a large NE annotated corpus, automatically generated
from WP articles. We followed a similar path to Nothman et
al.~\shortcite{Nothman:08} and broke down the process into four steps:

\begin{enumerate}
\item Classify WP articles into entity classes.
\item Parse WP and split articles into sentences.
\item Label named entities in the text.
\item Select the sentences for inclusion in the corpus.
\end{enumerate}

In this section, we describe how these steps were implemented. This section explains the general approach and its execution for English; Section \ref{sec:hun} describes how the idea is adapted to Hungarian.

%While the approach in essence is language-independent, the exact details have to be accommodated to the target language.

\subsection{Articles as Entities}
\label{dbpedia}

Many authors, such as Kazama and Torisawa \shortcite{Kazama:07} and Nothman et
al.~\shortcite{Nothman:08} used semi-supervised methods based on WP categories
and text to classify articles into NE types. To avoid the inevitable
classification errors, we obtain entity type information from the DBpedia
knowledge base \cite{Bizer:09}, which presents type, properties, home pages,
etc. information about pages in WP in structured form. With DBpedia we have
high precision information about entity types at the expense of recall: of the
3,903,467 English WP pages, 1,470,293 are covered by DBpedia (as of 18 March,
2012).

The types in DBpedia are organized into a class hierarchy, available as an OWL\footnote{http://www.w3.org/TR/owl-ref/}
ontology containing 320 frequent
entity categories, arranged into a taxonomy under the base class
\texttt{owl:Thing}. Most of the classes belong to the 6 largest
sub-hierarchies: \texttt{Person}, \texttt{Organisation}, \texttt{Event},
\texttt{Place}, \texttt{Species} and \texttt{Work}. The taxonomy is rather
flat: the top level contains 44 classes and there are several nodes with a
branching factor of 20.

The type of entities is extracted automatically from WP categories. However, the mapping between WP categories and classes in the DBpedia ontology is manually defined. This, together with the fact that the existence of the reference ontology prevents the proliferation of categories observable in WP \cite{Bizer:09}, ensures that type information in DBpedia can be considered gold quality.

From the available NER annotation standards we elected to use the CoNLL \cite{Tjong:03} NE types. It is not difficult to see the parallels between the DBpedia sub-hierarchies \texttt{Person}, \texttt{Organisation} and \texttt{Place} and the CoNLL NE types \texttt{PER}, \texttt{ORG} and \texttt{LOC}. The fourth category, \texttt{MISC} is more elusive; according to the CoNLL NER annotation guide\footnote{http://www.cnts.ua.ac.be/conll2003/ner/annotation.txt}, the sub-hierarchies \texttt{Event} and \texttt{Work} belong to this category, as well as various other classes outside the main hierarchies. 

While the correspondence described above holds for most classes in the
sub-hierarchies, there are some exceptions. For instance, the class
\texttt{SportsLeague} is part of the \texttt{Organisation} sub-hierarchy, but
according to the CoNLL annotation scheme, they should be tagged as
\texttt{MISC}.  To avoid misclassification,
%as well as the miscellaneous classes outside the main hierarchies, we have introduced a mapping mechanism. 
we created a file of DBpedia class--NE category mappings. Whenever an entity is evaluated, we look up its class and the ancestors of its class, and assign to it the category of the class that matches it most closely. If no match is found, the entity is tagged with \texttt{O}. 
%Since we take advantage of the inheritance hierarchy, the mapping list remains short: it contains only the root classes of the main hierarchies, exceptions like those mentioned above, and the various classes that belong to the MISC category according to the CoNLL annotation guideline.

As of version 3.7, the DBpedia ontology allows multiple superclasses, making a
directed acyclic
graph\footnote{http://blog.dbpedia.org/2011/09/11/dbpedia-37-released-including-15-localized-editions}. Since
selecting the right superclass, and hence, CoNLL tag, for classes with more
than one parent cannot be reliably done automatically, the class-to-category
mapping had to be determined manually. The only such class in version 3.7,
\texttt{Library}, can be traced back to both \texttt{Place} and
\texttt{Organisation}; its CoNLL tag is \texttt{LOC}.  Using the mapping thus
created, we compile a list that contains all entities in DBpedia tagged with
the appropriate CoNLL category.

We note here that our method can be trivially modified to work with any tagset
compatible with the DBpedia ontology (indeed, the DBpedia classes define a NE
tagset themselves), but we leave the exploration of these possibilities for
future work.

\subsection{Parsing Wikipedia}

WP is a rich source of information; in addition to the article text, a huge
amount of data is embedded in infoboxes, templates, and the category
structure. Our task requires only the links between the articles and the
article text. In addition to in-article links, our method takes advantage of
the redirect and interlanguage links, available as SQL dumps. The English
corpus is based on the WP snapshot as of January 15, 2011. The XML files were
parsed by the mwlib parser\footnote{http://code.pediapress.com}, the raw
text was tokenized by a modified version of the Punkt sentence and word
tokenizers \cite{Kiss:06}. For lemmatization we used the Wordnet Lemmatizer in
NLTK \cite{Bird:09}, and for part-of-speech tagging the HunPOS tagger
\cite{Halacsy:07}.

%Chunking was performed by HunChunk (CITE). Experiments have shown that HunPos obtains STAT f-score on the Brown corpus; HunChunk performs at around STAT, CITE?.

\subsection{Named Entity Labeling}

In order to automatically prepare sentences where NEs are accurately tagged,
two tasks need to be performed: identifying entities in the sentence and
tagging them with the correct tag. Sentences for which accurate tagging could
not be accomplished must be removed from the corpus. Our approach is based on
the work of Nothman et al.~\shortcite{Nothman:08}. The WP cross-references
found in the article text are used to identify entities. We assume that
individual WP articles describe NEs. A link to an article can then be
perceived as a mapping that identifies its anchor text with a particular NE.

The discovered entities are tagged with the CoNLL label assigned to them in
the entity list extracted from DBpedia. If the link target is not in the
entity list, or the link points to a disambiguation page, we cannot determine
the type of the entity, and tag it as UNK for subsequent removal from the
corpus. Links to redirect pages are resolved to point instead to the redirect
target, after which they are handled as regular cross-references. Finally, sentences with UNK links in them are removed from the corpus.

The following sub-sections describe how the method explained above can be
improved to increase precision, sentence coverage and to account for
peculiarities in the English orthography and the CoNLL guidelines.

\subsubsection{Non-entity Links}

Strictly speaking, our original assumption of equating WP articles with NEs is not valid: many pages describe common nouns (\textit{Book}, \textit{Aircraft}), calendar-related concepts (\textit{March 15}, \textit{2007}), or other concepts that fall outside the scope of NER. To increase sentence coverage, we modified the algorithm to prevent it from misclassifying links to these pages as unknown entities and discarding the sentence.

\begin{description}
\item[Common noun] links are filtered by POS tags; if a link contains no \texttt{NNP}s, it is ignored.
\item[Time expression] links require special attention, because dates and
  months are often linked to the respective WP pages. We circumvented this
  problem by compiling a list of calendar-related pages and adding them to the
  main entity list tagged with the CoNLL category \texttt{O}.
\item[Lowercase] links for entities referred to by common nouns, such as \textit{republic} to \textit{Roman Republic} are not considered NEs and are ignored.
\end{description}

\subsubsection{Unmarked Entities}

In a WP article, typically only the first occurrence of a particular entity is
linked to the corresponding page. Subsequent mentions are unmarked and often
incomplete -- e.g.~family names are used instead of full names. To account
for such mentions, we apply Nothman's \shortcite{Nothman:08} solution. For
each page, we maintain a list of entities discovered in the page so far and
try to associate capitalized words in the article text with these entities. We
augment the list with the aliases of every entity, such as titles of redirect
pages that target it, the first and last names in case of a \texttt{PER}
entity and any numbers in the name. If the current page is a NE, the
title and its aliases are added to the list as well; moreover, as WP usually
includes the original name of foreign entities in the article text, localized
versions of the title are also added to the list as aliases. Nothman's
solution used a trie to store the entity list, while we use a set, with more
alias types than what he used. We expect more precise tagging from our
slightly more rigorous solution.

\subsubsection{Special Cases}

%There are some special cases to the method, which are detailed below.

\begin{description}
\item[Derived words] According to the CoNLL guidelines, words derived from NEs are tagged as \texttt{MISC}. We complied with this rule by tagging each entity whose head is not a noun, as well as when the link's anchor text is not contained in the entity's name, as \texttt{MISC}. The most prominent example for such entities are nationalities, which can be linked to their home country, a \texttt{LOC}; e.g.~\textit{Turkish} to \textit{Turkey}. Our solution assigns the correct tag to these entities.
\item[First word in a sentence] As first words are always capitalized, labeling them is difficult if they are unlinked and not contained in the entity alias set. We base the decision on the POS tag of the first word: if it is \texttt{NNP}, we tag it as \texttt{UNK}; otherwise, \texttt{O}.
\item[Reference cleansing] Page titles and anchor texts may contain more than
  just the entity name. Personal titles are part of the entity name in WP, but
  not in CoNLL, and punctuation marks around the entity may become part of the
  link by mistake. We tag all punctuation marks after the entity name as
  \texttt{O}.
  
  To handle personal titles, we extracted a list from the WP page
  \textit{List of titles}, which contains titles in many languages. We
  manually removed all titles that also function as given names, such as
  \textit{Regina}. If a link to a \texttt{PER} or \texttt{UNK} entity, or an
  unlinked entity starts with, or consists solely of a title in the list, we
  tag the words that make up the title as \texttt{O}.
  % TODO 2 paragraphs?
\item[Incidental capitalization] Various non-\texttt{NNP} words in English are
  capitalized: names of months, the pronoun \textit{I}, and non-entity
  acronyms such as \textit{RSVP}. While the latter two types are unlikely to
  appear in WP text, we assembled a list of these words and tag them as
  \texttt{O} unless they are part of the alias set.
\end{description}

\subsection{Sentence Filtering}

As mentioned above, sentences with words tagged as \texttt{UNK} are
discarded. Furthermore, there are many incomplete sentences in the WP text:
image captions, enumerations items, contents of table cells, etc. On the one
hand, these sentence fragments may be of too low quality to be of any use in
the traditional NER task. On the other hand, they could prove to be invaluable
when training a NER tagger for User Generated Content, which is known to be
noisy and fragmented. As a compromise we included these fragments in the
corpus, but labelled them as ``low quality", so that users of the corpus can
decide whether they want to use them or not. A sentence is labelled as such if
it either lacks a punctuation mark at the end, or it contains no finite verb.

\section{Creating the Hungarian Corpus}
\label{sec:hun}

The procedure described in the previous section was used to generate the
Hungarian corpus as well. However, typological differences posed several
problems. In this section we describe the differences between the two
languages related to labeling NEs, and the changes they prompted in the
method.

\begin{table*}[ht]
\begin{center}
\begin{tabular}{lllll}
\hline \bf train & \bf test & \bf precision & \bf recall & \bf F-measure \\ \hline
Szeged NER & Szeged NER & 94.50 & 94.35 & 94.43 \\
huwiki & huwiki & 90.64 & 88.91 &  89.76 \\
huwiki & Szeged NER & 63.08 & 70.46 & 66.57 \\
Szeged NER with wikilists & Szeged NER & 95.48 & 95.48 & 95.48 \\
Szeged NER with wikitags & Szeged NER & 95.38 & 94.92 & 95.15 \\
\hline
\end{tabular}
\end{center}
\caption{\label{huresults} Hungarian results.}
\end{table*}

\subsection{Parsing the Hungarian Wikipedia}

Although Hungarian is reckoned to be a less resourced language, and it is not
supported in NLTK, several high quality language processing tools have been
developed for Hungarian in recent years. For tokenization and sentence
segmentation we used an in-house statistical tool tailored for Hungarian. It
has been trained on the largest manually annotated Hungarian corpus
\cite{Csendes:04}, and it handles the peculiarities of Hungarian orthography,
such as the periods placed after numbers in date expressions. Lemmatization
was performed by HunMorph \cite{Tron:05} and HunDisambig, an in-house
disambiguator to select the right analysis based on the word context.

%Although HunPos, as its name implies, is fully capable of POS tagging Hungarian text, we opted for Hunmorph (CITE), a morphological analyzer that, apart from PAS tagging, also lemmatizes the words. Hundisambig (CITE) was used to select the right analysis based on the word context\footnote{Incidentally, Hundisambig relies on HunPos for its HMM model.}.

For the most part Hungarian expresses grammatical elements within a word form
using affixes. HunMorph outputs KR-codes \cite{Kornai:04}, which, in addition
to the POS category, also include inflectional information, making it much
better suited to agglutinative languages than Penn Treebank POS tags. One
shortcoming of the KR-code is that it does not differentiate between common
and proper nouns. Since in Hungarian only proper nouns are capitalized, we can
usually decide whether a noun is proper based on the initial letter. However,
this rule can not be used if the noun is at the beginning of a sentence, so
sentences that begin with nouns have been removed from the corpus.

\subsection{Named Entity Labeling in Hungarian}

For well-resourced languages, DBpedia has internationalized chapters, 
but not for Hungarian. Instead, the Hungarian entity list comprises of the
pages in the English list that have their equivalents in the Hungarian WP. Two
consequences follow.  First, in order to identify which pages denote entities
in the Hungarian WP, an additional step is required, in which the Hungarian
equivalents of the English pages are added to the entity list. The English
titles are retained because (due to the medium size of the Hungarian WP)
in-article links sometimes point to English articles.

Second, entities without a page in the English WP are absent from the entity
list. This gives rise to two potential problems. One is that compared to
English, the list is relatively shorter: the entity/page ratio is 12.12\%, as
opposed to the 37.66\% of the English WP. The other, since mostly Hungarian
people, places and organizations are missing, a NER tagger that takes the
surface forms of words into account might be mislead as to the language model
of entity names. To overcome these problems, the list has to be extended with
Hungarian entity pages that do not have a corresponding English page. We leave
this for future work.

To annotate our corpus with NE tags, we chose to follow the annotation
guidelines of the largest human-annotated NER corpus for Hungarian, the Szeged
NER corpus \cite{Szarvas:06}. It is similar to CoNLL standards: contains
newswire texts, comprises ca.~200,000 tokens, and is annotated with NE class
labels in line with the CoNLL annotation scheme. However, the convention of
what constitutes a NE is slightly different for Hungarian.

\subsubsection{Special cases}

The Szeged NER guideline relies heavily on the rules of capitalization to
decide which words should be marked as NEs. The following concepts are not
proper nouns in Hungarian, and thus are not considered as NEs: names of
languages, nationalities, religions, political ideologies; adjectives derived
from NEs; names of months, days, holidays; names of special events and wars.

%\begin{itemize}
%\item names of languages, nationalities, religions, political ideologies;
%\item adjectives derived from a NE;
%\item names of months, days, holidays;
%\item names of special events and wars.
%\end{itemize}

There is another special case in Hungarian: unlike in English, the number of
compound words is quite large, and NEs can also be subject to compounding. In
this case the common noun following the NE is joined with a hyphen, so they
constitute one token. However, the joint common noun can modify the original
sense of NE, depending on the semantics of the common noun. For example in
the compound \textit{Nobel-díj} [`Nobel Prize'] the common noun changes the
labeling from \texttt{PER} to \texttt{MISC}, while
%\textit{SAP-szoftvercég} ['SAP software company'] the common noun does not
%change the labeling, but
in the case of the compound \textit{WorldCom-botrány} [`WorldCom scandal'] the NE tag changes from \texttt{ORG} to \texttt{O}. The solution to this problem is not obvious, and needs more investigation.

%Inflections of acronyms and foreign names ending with non-pronounced vowel have similar surface forms to the aforementioned compounds, e.g.~\textit{MTI-t} ['MTI'+ACC], \textit{Shakespeare-rel} ['with Shakespeare']. It is important to distinguish these types of hyphenated NEs, because inflections do not change NE labeling in contrast to some type of compounds. Due to the surface similarity of these phenomena the solution of this problem is not obvious, and needs more investigation. 

\section{Data Description}
\label{sec:data}

The corpora are available under the Creative Commons Attribution-Sharealike
3.0 Unported License (CC-BY-SA), the same license under which the text of WP
is released. The data files can be freely downloaded from \texttt{http://hlt.sztaki.hu}. The corpora
will also be distributed through the META-SHARE network, which is an open,
distributed facility for exchanging and sharing resources, and is one of the
lines of action of META-NET, a Network of Excellence funded by the European
Commission.

The files are in multitag format. Content lines are tab separated; there is one column for the tokens plus one column per tagset. Sentence boundaries are marked by empty lines. The linguistic features include the lemmatized form of the word and its POS tag. Two NE tags are included with each word: the most specific DBpedia category it belongs to and the CoNLL NE tag. While the NE tags can be considered as a ``silver standard", the linguistic features are provided on a ``best-effort" basis.

\begin{table*}[ht]
\begin{center}
\begin{tabular}{lllllll}
\hline  & \bf enwiki & \bf enwiki filtered & \bf CoNLL & \bf huwiki & \bf huwiki filtered  & \bf Szeged NER \\ \hline
token & 60,520,819 & 21,718,854 & 302,811 &  19,108,027 & 3,512,249  & 225,963\\
NE & 3,169,863 & 3,169,863 & 50,758 & 456,281 & 456,281  & 25,896\\
NE density & 5.23\% & 14.59\% & 16.76\% & 2.38\% & 12.99\%  & 11.46\%\\
\hline
\end{tabular}
\end{center}
\caption{\label{size} Corpus size and NE density.}
\end{table*}

\begin{table*}[ht]
\begin{center}
\begin{tabular}{lllll}
\hline \bf train & \bf test & \bf precision & \bf recall & \bf F-measure \\ \hline
CoNLL & CoNLL & 85.13 & 85.13 & 85.13 \\
enwiki & enwiki & 72.46 & 73.33 &  72.89 \\
enwiki & CoNLL & 56.55 & 49.77 & 52.94 \\
CoNLL with wikilists & CoNLL & 86.33 & 86.35 & 86.34 \\
CoNLL with wikitags & CoNLL & 85.88 & 85.94 & 85.91 \\
\hline
\end{tabular}
\end{center}
\caption{\label{enresults} English results.}
\end{table*}

\section{Evaluation}
\label{sec:eval}

Having the obvious advantages, an automatically generated corpus can not serve as a gold standard dataset. Then what can we do with silver standard corpora? They can be very useful for
improving NER in several ways: (a) for less resourced languages, they can
serve as training corpora in lieu of gold standard datasets; (b) they can
serve as supplementary or independent training sets for domains differing
from newswire; (c) they can be sources of huge entity lists, and (d) feature
extraction.

To evaluate our corpora we used a maximum entropy NE tagger \cite{Varga:07},
which was originally developed for labeling NEs in Hungarian texts, but can
be tuned for different languages as well. Corpus-specific features (e.g.~NP
chunks, WP links) were removed to get better comparability, so the feature set
consists of gazetteer features; sentence start and end position;
Boolean-valued orthographic properties of the word form; string-valued surface
properties of the word form; and morphological information.

We used the CoNLL standard method for evaluation. According to this, an automatic labeling is correct if it gives the same start and end position, and the same NE class as the gold standard. Based on this, precision and recall can be calculated, and the F-measure, as usual, the harmonic mean of these two values.  

\subsection{Wikipedia data}

Our automatic annotation process retains all of the WP sentences which
remained after our two-step filtering method, so sentences without NEs are
also included in the corpus. The rationale behind this is that we wanted to
reserve the original distribution of names in WP as much as possible. However,
after further investigation of the NE density in our corpora and gold standard
corpora, we decided not to include the sentences without NEs in evaluation
datasets.

Table \ref{size} summarizes the data regarding corpus size and NE density. The
English (enwiki) and the Hungarian WP (huwiki) corpora originally have the NE
density of 5.23\% and 2.38\%, respectively. In comparison to the gold standard
datasets (CoNLL, Szeged NER) these counts are quite low. It can be due to the
difference between domains: newswire articles usually contain more NEs,
typically ORG. The other reason might be that we discarded sentences
containing unidentified NEs (cf.~Section \ref{sec:create}).

\subsection{Experiments and results}

The English WP corpus was evaluated against itself and a manually annotated
English corpus. Since the filtered English WP corpus, containing only the
sentences with NEs, is still very large, our experiments were performed with a
sample of 3.5 million tokens, the size of our filtered Hungarian corpus,
divided into train and test sets (90\%-10\%).

For English cross-corpus evaluation the CoNLL-2003 corpus was chosen. As is
well known, training and testing across different corpora decreases
F-measure. Domain differences certainly affect NER performance, and the
different annotation schemes pose several compatibility problems. Nothman et
al.~\shortcite{Nothman:08} showed that each set of gold standard training data
performs better on corresponding test sets than on test sets from other
sources. The situation here is similar (see Table \ref{enresults} for
results): the NE tagger trained on WP does not achieve as high performance
tested against CoNLL test set (enwiki-CoNLL) as one trained on its own train
set (enwiki-enwiki).

WP-derived corpora can also be used for improving NER accuracy in other
ways. First, we collected gazetteer lists from the corpus for each NE
category, which improved the overall F-measure given to the NE tagger training
and testing on CoNLL dataset (CoNLL with wikilists). A second trial was
labeling the CoNLL datasets by the model trained on WP corpus, and giving
these labels as extra features to the next CoNLL train (CoNLL with
wikitags). Both methods result in improved F-measure on CoNLL test set.

Since in Hungarian NE tagging we followed the Szeged NER corpus annotation
guidelines, we performed the experiments on this dataset. Hungarian results
are similar to the English ones (see Table \ref{huresults}), the only
difference is that F-measures for Hungarian are significantly higher. This can
be due to the fact that the MISC category for Hungarian contains less types of
names, thus the inconsistency of this class is smaller (cf.~Section
\ref{sec:hun}). In contrast to the CoNLL corpus, the Szeged NER corpus was
accurately annotated with an inter-annotator agreement over 99\%.

Due to the quite good F-measure of training on our Hungarian train corpus and
testing on the corresponding test set, our Hungarian corpus can serve as a
training corpus to build NE taggers for non-newswire domains.

\section{Conclusion}
\label{sec:conclusion}

We have presented freely available NE tagged corpora for English and
Hungarian, fully automatically generated from WP. In contrast to the methods
used so far for automatic annotation of NEs in WP texts, we applied a new
approach, namely mapping DBpedia ontology classes to standard CoNLL NE tags,
and assigning them to WP entities. 
Following Nothman \shortcite{Nothman:08},
the process can be divided into four main steps: classifying WP articles into
entity classes; parsing WP and splitting articles into sentences; labeling NEs
in the text; and selecting sentences for inclusion in the corpus.

The huge amount of WP articles opens the possibility of building large enough
corpora for otherwise less resourced languages such as Hungarian. Due to the
particularities of Hungarian, some steps are slightly different, and special
linguistic phenomena pose several problems related to the NER task to solve.

Automatically generated corpora can be useful for improving NER in more
ways. We showed that gazetteer lists extracted from our corpora, and training
with extra features given by the model trained on our corpora, improve
F-measure. Moreover, our Hungarian corpus can serve as a training corpus for
more general domains than the classic newswire.

%később majd kirakjuk a scripteket is, és akár újra és újra frissíthető lesz

\section*{Acknowledgements}

This research was supported by OTKA grant no. 82333 and the CESAR project under the ICT Policy Support Programme (grant no. 271022). The authors are grateful to Attila Zséder for his work on Wikipedia parsing and to András Kornai for his insightful comments.

\begin{thebibliography}{}

%\bibitem[\protect\citename{Balasuriya \bgroup et al.\egroup}2009]{Balasuriya:09}
%Dominic Balasuriya, Nicky Ringland, Joel Nothman, Tara Murphy, James R. Curran
%\newblock 2009.
%\newblock Named Entity Recognition in Wikipedia.
%\newblock In: {\em Proceedings of the 2009 Workshop on the People's Web Meets NLP, ACL-IJCNLP}, pages 10--18.

\bibitem[\protect\citename{Bird et al.}2009]{Bird:09}
Steven Bird, Ewan Klein, Edward Loper.
\newblock 2009.
\newblock {\em Natural Language Processing with Python}.
\newblock O'Reilly Media Inc.

\bibitem[\protect\citename{Bizer \bgroup et al.\egroup}2009]{Bizer:09} 
Christian Bizer, Jens Lehmann, Georgi Kobilarov, Sören Auer, Christian Becker, Richard Cyganiak, Sebastian Hellmann.
\newblock 2009.
\newblock DBpedia -- A Crystallization Point for the Web of Data. 
\newblock In: {\em Journal of Web Semantics: Science, Services and Agents on the World Wide Web}, Issue 7, pages 154--165.

\bibitem[\protect\citename{Bunescu and Pasca}2006]{Bunescu:06}
B. Bunescu and M. Pasca.
\newblock 2006.
\newblock Using encyclopedic knowledge for named entity disambiguation.
\newblock In: {\em Proceedings of the 11th Conference of the European Chapter of the Association for Computational Linguistics}, pages 9--16.

\bibitem[\protect\citename{Csendes \bgroup et al.\egroup}2004]{Csendes:04} 
Dóra Csendes, János Csirik, Tibor Gyimóthy.
\newblock 2004.
\newblock The Szeged Corpus: A POS tagged and Syntactically Annotated Hungarian Natural Language Corpus. 
\newblock In: {\em Proceedings of TSD 2004}, vol. 3206, pages 41--49.

\bibitem[\protect\citename{Cucerzan}2007]{Cucerzan:07}
S. Cucerzan.
\newblock 2007.
\newblock Large-scale named entity disambiguation based on Wikipedia data.
\newblock In: {\em Proceedings of the 2007 Joint Conference on Empirical Methods in Natural Language Processing and Computational Natural Language Learning}.
\newblock Prague, Czech Republic, June 2007. pages 708--716.

\bibitem[\protect\citename{Grishman and Sundheim}1996]{Grishman:96}
Ralph Grishman and B. Sundheim.
\newblock 1996.
\newblock Message Understanding Conference -- 6. 
\newblock In: {\em Proc. International Conference on Computational Linguistics}.

\bibitem[\protect\citename{Halácsy \bgroup et al.\egroup}2007]{Halacsy:07}
P. Halácsy, A. Kornai and Cs. Oravecz.
\newblock 2007.
\newblock Hunpos -- an open source trigram tagger. 
\newblock In: {\em Proceedings of the 45th Annual Meeting of the Association for Computational Linguistics}, pages 209--212.

\bibitem[\protect\citename{Kazama and Torisawa}2007]{Kazama:07}
Jun'ichi Kazama and Kentaro Torisawa.
\newblock 2007.
\newblock Exploiting Wikipedia as External Knowledge for Named Entity Recognition.
\newblock In: {\em Proceedings of the 2007 Joint Conference on Empirical Methods in Natural Language Processing and Computational Natural Language Learning}, pages 698--707.

\bibitem[\protect\citename{Kiss and Strunk}2002]{Kiss:06}
Tibor Kiss, Jan Strunk.
\newblock 2006.
\newblock Unsupervised Multilingual Sentence Boundary Detection. 
\newblock In: {\em Computational Linguistics}, 32 (4): pages 485--525.

\bibitem[\protect\citename{Kornai \bgroup et al.\egroup}2004]{Kornai:04}
András Kornai, Péter Rebrus, Péter Vajda, Péter Halácsy, András Rung, Viktor Trón.
\newblock 2004.
\newblock Általános célú morfológiai elemző kimeneti formalizmusa (The output formalism of a general-purpose morphological analyzer).
\newblock In: {\em Proceedings of the 2nd Hungarian Computational Linguistics Conference}.

\bibitem[\protect\citename{Medelyan  \bgroup et al.\egroup}2009]{Medelyan:09}
Olena Medelyan, David Milne, Catherine Legg, and Ian H. Witten.
\newblock 2009.
\newblock Mining meaning from Wikipedia.
\newblock {\em International Journal of Human-Computer Studies}, 67: 716--754.

\bibitem[\protect\citename{Nadeau  \bgroup et al.\egroup}2006]{Nadeau:06} 
David Nadeau, Peter D. Turney and Stan Matwin.
\newblock 2006.
\newblock Unsupervised named entity recognition: Generating gazetteers and resolving ambiguity. 
\newblock In: {\em Proceedings of the 19th Canadian Conference on Artificial Intelligence}, volume 4013 of {\em LNCS}, pages 266--277.

\bibitem[\protect\citename{Nothman \bgroup et al.\egroup}2008]{Nothman:08}
Joel Nothman, James R. Curran, and Tara Murphy.
\newblock 2008.
\newblock Transforming Wikipedia into Named Entity Training Data.
\newblock In: {\em Proceedings of the Australasian Language Technology Workshop}, Vol 6., pages 124--132.

\bibitem[\protect\citename{Nothman \bgroup et al.\egroup}2009]{Nothman:09}
Joel Nothman, Tara Murphy and James R. Curran.
\newblock 2009.
\newblock Analysing Wikipedia and Gold-Standard Corpora for NER Training.
\newblock In: {\em Proceedings of the 12th Conference of the European Chapter of the ACL}, pages 612--620.

\bibitem[\protect\citename{Richman and Schone}2008]{Richman:08}
Alexander E. Richman and Patrick Schone.
\newblock 2008.
\newblock Mining Wiki Resources for Multilingual Named Entity Recognition.
\newblock In: {\em Proceedings of ACL-08: HLT}, pages 1--9.

\bibitem[\protect\citename{Szarvas \bgroup et al.\egroup}2006]{Szarvas:06}
György Szarvas, Richárd Farkas, András Kocsor.
\newblock 2006.
\newblock A highly accurate Named Entity corpus for Hungarian. 
\newblock In: {\em Proceedings of International Conference on Language Resources and Evaluation}.

\bibitem[\protect\citename{Tjong Kim Sang}2002]{Tjong:02}
Erik F. Tjong Kim Sang.
\newblock 2002.
\newblock Introduction to the CoNLL-2002 shared task: Language-independent named entity recognition.
\newblock In: {\em Proceedings of the 6th Conference on Natural Language Learning}, pages 1--4, Taipei, Taiwan.

\bibitem[\protect\citename{Tjong Kim Sang and De Meulder}2003]{Tjong:03}
Erik F. Tjong Kim Sang and Fien De Meulder.
\newblock 2003.
\newblock Introduction to the CoNLL-2003 shared task: Language-independent named entity recognition. 
\newblock In: {\em Proceedings of the 7th Conference on Natural Language Learning}, pages 142--147, Edmonton, Canada.

\bibitem[\protect\citename{Toral and Mu\~noz}2006]{Toral:06}
A. Toral and R. Mu\~noz.
\newblock 2006.
\newblock A proposal to automatically build and maintain gazetteers for Named Entity Recognition by using Wikipedia.
\newblock In: {\em EACL 2006}.

\bibitem[\protect\citename{Trón \bgroup et al.\egroup}2005]{Tron:05}
Viktor Trón, György Gyepesi, Péter Halácsy, András Kornai, László Németh, Dániel Varga.
\newblock 2005.
\newblock Hunmorph: open source word analysis.
\newblock In: {\em Proceedings of the ACL 2005 Workshop on Software}. 

\bibitem[\protect\citename{Varga and Simon}2007]{Varga:07}
Dániel Varga and Eszter Simon.
\newblock 2007.
\newblock Hungarian named entity recognition with a maximum entropy approach.
\newblock {\em Acta Cybernetica}, 18: 293--301.

\bibitem[\protect\citename{Zaragoza \bgroup et al.\egroup}2007]{Zaragoza:07}
H. Zaragoza and J. Atserias and M. Ciaramita and G. Attardi.
\newblock 2007.
\newblock Semantically Annotated Snapshot of the English Wikipedia v.1 (SW1).
\newblock \url{http://www.yr-bcn.es/semanticWikipedia}


\end{thebibliography}

%\appendix
%\section{DBpedia to CoNLL Mapping}
%\label{sec:appa}
%
%\begin{table*}[ht]
%\begin{center}
%\begin{tabular}{llll}
%\hline DBpedia & \bf CoNLL & \bf DBpedia & \bf CoNLL \\ \hline
%Person & PER & Library & LOC \\
%Place & LOC & MeanOfTransportation & MISC \\
%Organization & ORG & PeriodicalLiterature & ORG \\
%Organisation & ORG & ProgrammingLanguage & MISC \\
%Award & MISC & Project & MISC \\
%EthnicGroup & MISC & SportsLeague & MISC \\
%Event & MISC & SportsTeamSeason & O \\
%Holiday & MISC & Weapon & MISC \\
%Ideology & MISC & Work & MISC \\
%Language & MISC & & \\
%\hline
%\end{tabular}
%\end{center}
%\end{table*}

\end{document}

% Jobban ki kell hangsúlyozni az introductionben, és a related worksben, hogy szabadon elérhető training korpuszt építettünk
% És azt, hogy Hungarian middle, English nagy.
