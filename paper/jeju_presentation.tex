% $Header: /home/vedranm/bitbucket/beamer/solutions/generic-talks/generic-ornate-15min-45min.en.tex,v 90e850259b8b 2007/01/28 20:48:30 tantau $

\documentclass[utf8x]{beamer}

\mode<presentation>
{
%  \usetheme[height=7mm]{Rochester}
  \usetheme{Rochester}
  \useinnertheme{rounded}
%  \useoutertheme{infolines}
  % or ...

  \setbeamercovered{transparent}
}
\usefonttheme[onlymath]{serif}

\usepackage[english]{babel}
\usepackage[utf8x]{inputenc}

\usepackage{times}
\usepackage[T1]{fontenc}
% Or whatever. Note that the encoding and the font should match. If T1
% does not look nice, try deleting the line with the fontenc.

%% to draw trees without jpg figures
%\usepackage{synttree}
% For code blocks
\usepackage{algorithm}  % Conflict w/ synttree
\usepackage{algorithmic}
% for sane tabular handling
\usepackage{array}
% For strikethrough (normalem keeps it as emph)
\usepackage[normalem]{ulem}

\usepackage{mathrsfs}
\usepackage{bm}  % For bold math

\newcommand{\vitem}{\vfill \item}

\title % (optional, use only with long paper titles)
{Reinforcement Learning End of Term Presentation}

%\subtitle
%{Presentation Subtitle} % (optional)

\author % (optional, use only with lots of authors)
{Eszter ~Simon, David Márk ~Nemeskey}

% TODO: RILHAS
\institute[Computer and Automation Research Institue of the Hungarian Academy of Sciences] % (optional, but mostly needed)
{
  Data Mining and Search Research Group\\
  MTA SZTAKI
}
% - Use the \inst command only if there are several affiliations.
% - Keep it simple, no one is interested in your street address.

\date % (optional)
{2012.07.12.}

% \subject{Talks}
% This is only inserted into the PDF information catalog. Can be left
% out. 



% If you have a file called "university-logo-filename.xxx", where xxx
% is a graphic format that can be processed by latex or pdflatex,
% resp., then you can add a logo as follows:

% \pgfdeclareimage[height=0.5cm]{university-logo}{university-logo-filename}
% \logo{\pgfuseimage{university-logo}}

\AtBeginSection[]
{
   \begin{frame}
       \frametitle{Outline}
       \tableofcontents[currentsection]
   \end{frame}
}


% If you wish to uncover everything in a step-wise fashion, uncomment
% the following command: 
%\beamerdefaultoverlayspecification{<+->}


\begin{document}

% TODO: kell outline?
\begin{frame}{Outline}
  \titlepage
\end{frame}

\section{Creating the English Corpus}
\subsection*{Motivation}
\begin{frame}{Blabla}
\end{frame}
\subsection{Articles to Entities}
\begin{frame}{Problem Statement}
  In many real world applications the problem is to predict outputs with a nontrivial structure given some inputs.
  \vfill
  Example: \textit{\small It was love at first sight.}
  \vfill
\end{frame}

\section{Creating the Hungarian Corpus}
\begin{frame}{The Hungarian Corpus}
  The same algorithm described thus far can be applied to Hungarian with
  minimal changes.
  \vfill
  Differences between Hungarian and English:
  \begin{itemize}
  \vitem Language processing tools % different -> flexible framework
  \vitem Wikipedia size
  \vitem DBpedia coverage
  \vitem NE conventions
  \end{itemize}
\end{frame}

\section{Conclusions}
\begin{frame}{Results -- cont.}
  Single hold-out vs. cross-validation:
  \begin{itemize}
  \vitem Single hold-out is biased
  \vitem cross-validation gives more stable results
  \end{itemize}
  \vfill
\end{frame}

\end{document}


